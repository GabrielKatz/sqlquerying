\documentclass[12pt,a4paper]{report}

\setlength{\topmargin}{0 cm}
\usepackage{lmodern}
\usepackage{times}
\usepackage{float}
\usepackage{rotating}
\usepackage[german]{babel}
\usepackage{graphics}
\usepackage{listings}
\makeatletter
\newcommand{\lstuppercase}{\uppercase\expandafter{\expandafter\lst@token
                           \expandafter{\the\lst@token}}}
\newcommand{\lstlowercase}{\lowercase\expandafter{\expandafter\lst@token
                           \expandafter{\the\lst@token}}}
\makeatother
\lstdefinestyle{SQL}{basicstyle=\ttfamily,
                        keywordstyle=\lstuppercase,
                        emphstyle=\itshape,
                        showstringspaces=false,
                        numbers=left
                        }
                        
\graphicspath{ {../img/} }
\usepackage[table,dvipsnames]{xcolor}
\usepackage{caption}
\usepackage{subcaption}
\usepackage{ntheorem}
\usepackage{pifont}
\theoremstyle{break}
\theorembodyfont{\normalfont}
\theoremprework{\bigskip\hrule\leavevmode\nopagebreak}
\theorempostwork{\nopagebreak\hrule\leavevmode}
\newtheorem{exercise}{Aufgabe}[chapter]
\theoremstyle{plain}
\theoremprework{\bigskip\hrule\leavevmode\nopagebreak}
\theorempostwork{\nopagebreak\hrule\leavevmode}
\newtheorem{proof}{Satz}[chapter]
\renewcommand{\labelenumii}{\arabic{enumi}.\arabic{enumii}.}
\newcommand{\algostep}[2]{\noindent\parbox{4cm}{\scalebox{0.5}{\includegraphics{#1}}}
  \hfill
  \parbox{7cm}{#2}
  \vskip -5mm
}
\newcommand{\ucl}{\ding{56}}
\newcommand{\ign}{\rowcolor[gray]{0.5}}
\newcommand{\usd}[1]{{\cellcolor{SkyBlue}#1}}
\newcommand{\sel}[1]{{\cellcolor{LimeGreen}#1}}
\newcommand{\byp}[1]{{\cellcolor{LimeGreen!10}#1}}
\newcommand{\matrixstep}[2]{\noindent\parbox{6cm}{\scriptsize #1}
  \hfill
  \parbox{7cm}{\small #2}
  \vskip 2mm
}

\title{Relationale Datenabfragen mit SQL}
\author{Gabriel Katz}

\begin{document}
\maketitle
\tableofcontents
\chapter{Einleitung}

\section{Worum geht es hier?}
Eines der wichtigsten Zwecke von Computer ist die Speicherung von Daten. Wenn eine Menge von organisierten
Informationen dauerhaft auf einem zentralen Computer gespeichert werden, und diese Daten leicht abgefragt, 
erg\"{a}nzt, bearbeitet und gel\"{o}scht werden kann, ist von einer \textbf{Datenbank} die Rede. So ist zum Beispiel
ein Notizblock (egal ob in Papierform oder in digitaler, wie zum Beispiel Notes oder OneNote) eine stark 
eingeschr"ankte Datenbank. Um relevante Daten zu finden, muss man eventuell Seite f"ur Seite durchgehen. Beim 
Notizblock aus Papier ist auch die "Uberarbeitung der Daten eingeschr"ankt, da man z.B. auf der Seite nur begrenzt
Platz hat. Daher stellt sich die Frage, welche besser zum Bearbeiten geeigneten Wege es gibt, Daten auf dem Computer
zu speichern.
\iffalse
Es gibt 
verschiedene Arten von Datenbanken. Die relationalen Datenbanken werden wir im n\"{a}chsten Abschnitt 
genauer betrachten. \textbf{Hierarchische} Datenbanksysteme stammen noch von der Zeit vor den relationalen 
Datenbanken. Dort sind die Daten in einer Baumstruktur abgelegt.  Zur Zeit gibt es einen starken Trend zu sogenennten 
\textbf{NoSQL}-Datenbanken. Dies sind nichtrelationale Datenbanken, bei welchen die St\"{a}rken in der Verteilung 
der Dateien und den Berechnungen auf verschiedene Rechner liegt. Es gibt verschiedene Arten, wie solche Daten 
abgelegt werden k\"{o}nnen: Dokumentenorientiert, objektorientiert und Key-Value-Datenbanken sind die 
verbreitetesten dieser Arten.
\fi
\section{Relationale Datenbanken}
Das Konzept f\"{u}r relationale Datenbanken basiert auf einem Paper von Dr. E. F. Codd aus dem Jahre 1970 
namens ''A Relational Model of Data for Large Shared Data Banks''. Dort schlug er vor, Daten in einer Menge von 
\textbf{Tabellen} darzustellen. Dies Tabellen haben einen Namen und mehrere \textbf{Spalten}. Eine Spalte einer 
Tabelle ist durch seinen Namen, seinen Datentyp, und ob sie obligatorisch oder fakultativ ist, definiert. Datentypen 
dr\"{u}cken aus, welche Art von Daten in der Spalte gespeichert werden kann. Beispielsweise hat eine Spalte, in 
welcher ein Datum gespeichert wir einen anderen Datentypen wie eine Spalte mit Zahl oder wie eine Spalte mit Text. 
Wir werden die Datentypen in SQL im n\"{a}chsten Kapitel etwas mehr im Detail betrachten. \\
Die Daten werden 
dann in den \textbf{Zeilen} der Tabelle gespeichert. Jede Zeile der Tabelle enth\"{a}lt mindestens in allen 
obligatorischen Spalten Daten. Normalerweise ist die Anzahl der Spalten der Tabelle (des Datenschemas) 
\"{u}berschaubar, wogegen die Anzahl der Zeilen riesig sein kann. \\
Eine Tabelle in einer relationalen Datenbank hat immer ein Merkmal, an welchem die Zeilen der Tabellen eindeutig
identifiziert werden k�nnen. Dieses Merkmal wird \textbf{Prim\"arschl\"ussel (PK)} genannt. Dieser Prim\"arschl\"ussel 
besteht h\"aufig aus einer einzelnen Tabellenspalte, doch er kann auch aus mehreren Tabellenspalten bestehen.
In diesem Fall spricht man von einem \textbf{zusammengesetzten} Schl\"ussel. H\"aufig ist der Prim\"arschl\"ussel eine
Spalte, welche nicht eigentliche Daten enth\"alt, sondern nur ein Kennzeichen.\\
Die Kernidee hinter den relationalen Datenbanken, ist, dass diese Tabellen verbunden sind. Doch wie sieht eine 
derartige Verbindung aus? Hier kommen die sogenannten \textbf{Fremdschl\"ussel (FK)} ins Spiel. Ein Fremdschl\"ussel
besteht aus einer oder mehreren Spalten einer Tabelle, welche den Prim\"arschl\"ussel einer anderen Tabelle 
referenzieren. Da zusammengesetzte Fremdschl\"ussel in der Praxis selten auftreten, werden wir diese in den Beispielen
nicht behandeln.\\
\begin{table}
    \begin{tabular}{|llll|}
    \hline
    CustomerId (PK) & FirstName & LastName & Email               \\ \hline
    1               & Hans      & Muster   & hmuster@example.com \\
    2               & Anja      & Tester   & atester@example.com \\
    3               & Ferdinand & Meier    & fmeier@example.com  \\ \hline
    \end{tabular}

    \caption{Customer}
    \label{tbl:firstExampleStart}
	\begin{tabular}{|lll|}
    \hline
    ServiceId (PK) & Name                      & Duration \\ \hline
    1              & Waschen, Schneiden        & 30:00    \\
    2              & Waschen, Schneiden, F"onen & 45:00    \\
    3              & T\"onen                     & 90:00    \\ \hline
    \end{tabular} 
    \caption{Service}
    \begin{tabular}{|llll|}
    \hline
    BookingId &ServiceFk  & CustomerFk  & StartDate        \\ 
       (PK) &(FK(Service.ServiceId)) & (FK(Customer.CustomerId)) &  \\ \hline
    1 &1                  & 3                   & 21.11.2014 10:30 \\
    2 &1                  & 1                   & 24.11.2014 08:30 \\
    3 & 3                  & 2                   & 23.11.2014 15:00 \\ \hline
    \end{tabular}
    \caption {Booking}
    \label{tbl:firstExampleEnd}
\end{table}

Tabellen~\ref{tbl:firstExampleStart} bis~\ref{tbl:firstExampleEnd} zeigen ein Beispiel f\"ur eine einfache Datenbank, wie 
sie in einem Buchungssystem eines Friseursalons vorstellbar ist. In der Tabelle $Customer$ sind die Kundeninformation 
aufgelistet. \"Andert sich beispielsweise die Emailadresse eines Kunden, muss diese lediglich an einer Stelle, n\"amlich 
in der entsprechenden Spalte der $Customer$-Tabelle angepasst werden. Wir sehen hier, dass der Prim\"arschl\"ussel, 
das Feld $CustomerId$, keine eigentlichen Information zum Benutzer enth\"alt. Doch wieso wurde dieses Feld 
eingef\"uhrt? Betrachten wir die Alternativen f\"ur den Prim\"arschl\"ussel: Wenn die Spalte $Email$ Prim\"arschl\"ussel
w\"are, entst\"anden zwei Nachteile. Einerseits k\"onnten die Kunden ihre Mailadresse nicht mehr wechseln, da sonst
das eindeutige Merkmal der Kundenzeile verloren gehen w\"urde. Andererseits k\"onnen keine zwei Benutzer die gleiche
Mailadresse teilen. Die Spalten $FirstName$ und $LastName$ sind nicht eindeutig, und k"onnen somit nicht als 
Prim\"arschl\"ussel (auch zusammengenommen nicht) gew\"ahlt werden. Es g\"abe noch die M"oglichkeit, die 
Kombination $FirstName$, $LastName$ und $Email$ als Prim\"arschl\"ussel zu w\"ahlen, doch zugunsten der 
Mutabilit\"at der Daten entscheidet man meistens gegen derart komplexe Prim\"arschl\"ussel und f\"uhrt lieber eine 
weitere Zeile ein.

Bei der Tabelle $Service$ verh\"alt es sich gleich wie bei der $Customer$-Tabelle: Da die anderen Spalten sich \"andern 
k\"onnen, und eventuell nicht eindeutig sind, wurde eine zus\"atzliche Spalte f\"ur den Prim\"rschl\"ussel eingef\"hrt,

Die $Booking$-Tabelle hat zwei Fremdschl\"ussel: Die Spalte $ServiceFk$ referenziert die Spalte $ServiceId$ der Tabelle 
$Service$, und die Spalte $CustomerFk$ referenziert die Spalte $CustomerId$ der Tabelle $Customer$. Der erste Eintrag
in der Booking-Tabelle bedeutet also, dass Ferdinand Meier (der Kunde mit der $CustomerId$ 3) am 21. November 2014
um 10:30 einen 45-minutigen Termin zum Haare waschen und schneiden hat (der Service mit der $ServiceId$ 1).

In diesem 
Beispiel enden Spalten, welche andere Tabellen referenzieren auf $-Fk$, und Prim\"arschl\"ussel auf $-Id$. Es handelt 
sich hierbei um eine Namenskonvention des Autoren. Andere Namen sind durchaus denkbar.
 
 \begin{exercise}\label{ex:changeFK}
Was bedeutet es, wenn beim Eintrag mit der $BookingId$ 2 der $ServiceFK$ von 1 auf 2 ge\"andert wird?
\end{exercise}
 \begin{exercise}\label{ex:noComposedKey}
Wieso k\"onnen die Spalten $ServiceId$ und $CustomerId$ keinen zusammengesetzten Prim\"arschl\"ussel f\"ur die 
Tabelle $Booking$ bilden?
\end{exercise}

Wir betrachten noch ein Beispiel f"ur eine Datenbank. Tabellen Tabellen~\ref{tbl:secondExampleStart} 
bis~\ref{tbl:secondExampleEnd}  sind Teil einer Musikbibliotheksdatenbank. In diesem Beispiel kommt ein 
zusammengesetzter Prim"arschl"ussel vor. Da der Track eines Albums durch das Album und der Tracknummer eindeutig 
gegeben ist, und die Tracknummer und das Album sich nie mehr "andert, eignet sich diese Kombination hervorragend
als zumsammengesetzten Prim"arschl"ussel.

Ein weiteres h"aufiges Muster, dass bei relationalen Datenbanken vorkommt, ist bei der Genre-Tabelle zu finden. Wir sehen,
dass die $ParentGenreFk$ die eigene Tabelle referenziert. Im Beispiel ist $Punk Rock$ ein Subgenre von $Rock$. Wir stellen 
ausserdem fest, dass gewisse Genres einen $NULL$-eintrag beim Feld $ParentGenreFk$ haben. Dies ist die Notation, welche
besagt, dass dieses Feld f"ur einen Eintrag nicht verwendet wird. Eine Zeile kann nur in den Spalten $NULL$-eintr"age haben,
in welchem
 \begin{exercise}\label{ex:NULL}
K"onnte in der Spalte $AlbumFk$ der Tabelle $Track$ ein $NULL$-eintrag stehen? Wieso bzw. wieso nicht?
\end{exercise}
\begin{table}
    \begin{tabular}{|llll|}
    \hline
    AlbumId (PK) & ArtistName & AlbumName  &   GenreFk  (FK)        \\ \hline
    1               & Red Hot Chilli Peppers      & Californication & 1 \\
    2               & The Ramones      & Rocket To Russia  & 2\\
    3               & The Beastie Boys & Ill Communication & 3\\ \hline
    \end{tabular}

    \caption{Album}
    \label{tbl:secondExampleStart}
	\begin{tabular}{|llll|}
    \hline
    AlbumFk (PK, FK) & TrackId   (PK)  & Name & Duration \\ \hline
    3		& 1 & Sure Shot       & 3:20    \\
    1		& 1 & Around the World       & 3:59    \\
    2       	& 2 & Rockaway Beach  &  2:07    \\
    1         	& 4 & Otherside        & 4:15    \\ \hline
    \end{tabular} 
    \caption{Track}
    \label{tbl:tracks}
    \begin{tabular}{|lll|}
    \hline
    GenreId(PK) &ParentGenreFk (FK) & Name        \\ \hline
    1 &NULL                  & Rock \\
    2 &1                  & Punk Rock \\
    3 & NULL  & Hip-Hop\\ \hline
    \end{tabular}
    \caption {Genre}
    \label{tbl:secondExampleEnd}
\end{table}

\newpage
\chapter{Relationale Datenbanken in SQL}
\section{Was ist SQL?}
Dr. E. F. Codd, der wie im letzten Kapitel beschrieben die relationale Datenbanken erfunden hat, hat auch eine Sprache 
namens DSL/Alpha entwickelt, um Daten in relationellen Tabellen zu bearbeiten. Nach einigen Weiterentwicklungen durch 
Donald D. Chamberlin und Raymond F. Boyce bei IBM Research entstand dann die Sprache SEQUEL (kurz f"ur Standard 
English Query Language), welche dann aus markenrechtlichen Gr"unden in SQL umbenannt werden musste. SQL ist seit 
1986 von ANSI (American National Standard Institute) standardisiert, und dieser Standard wurde regelm"assig aktualisiert, 
so dass heutzutage auch zum Beispiel die Integration und Abfrage von XML in SQL-Datenbanken standardisiert ist.

Im Gegensatz zu C, C\#, Javascript oder Java ist SQL keine Allzwecksprache. Sie dient lediglich der Bearbeitung und Abfrage
von Relationen Daten. Mit SQL werden lediglich notwendige Ein- und Ausgaben beschrieben. Wie die Daten gespeichert, bzw.
wiedergegeben werden, ist Aufgabe der Datenbankengine. Diese optimiert die Ausf"uhrung der Datenbankabfrage, so dass 
die Daten m"oglichst effizient abgefragt werden.

SQL an sich ist keine vollst"andige Programmiersprache. Es gibt jedoch verschieden Erweiterungen, wie PL/SQL von Oracle,
MySQL oder T-SQL von Microsoft, durch welche prozedural mit SQL gearbeitet werden kann. Viele
Datenbankanwendungen werden aber nicht direkt mit SQL entwickelt, sondern mit einem Toolkit oder einer API aus einer 
Allzwecksprache heraus. Beispiele daf"ur sind JDBC f"ur Java oder das Entity Framework f"ur C\# und Visual Basic. 
\section{Datentypen}
In SQL gibt es, "ahnlich wie in Programmiersprachen, verschiedene Datentypen. Es gibt eine Vielzahl von komplexeren Datentypen,
welche f"ur spezifische Zwecke, wie zum Beispiel f"ur Bilder und XML-Dateien existieren. Doch da wir uns vor allem f"ur die
Datenabfrage interessieren, richten wir unseren Fokus auf drei verschiedene Arten von Datentypen: Zahlwerte, temporale Daten und
Zeichenketten.  

\subsection{Zahlwerte}
Ein \textbf{BIT} kann nur entweder 0 (false) oder 1 (true) sein. Dieser Datentyp ist ideal f"ur Antworten auf Ja/Nein-Fragen, welche nicht
nur in der Informatik h"affig auftreten.
Ein \textbf{INT} ist eine Ganzzahl. Es gibt verschiedene Variationen davon f"ur verschiedene gr"ossen (\textbf{SMALLINT}, \textbf{BIGINT}, etc.),
doch das Grundprinzip ist das Gleiche. In den vorhergehenden Beispielen wurden $INT$s vor allem f"ur Schl"ussel verwendet.
Ein \textbf{FLOAT} ist eine Gleitkommazahl. Dies heisst, dass diese Zahl fast beliebig gross oder klein sein kann.
Ein \textbf{DECIMAL(p,s)} ist eine Zahl mit $p$ Ziffern insgesamt, davon $s$ nach dem Komma. Dies heisst, dass z.B. $41.46$ als 
\textbf{DECIMAL(4,2)} gespeichert werden kann, jedoch wird $31.8734$ auf $31.87$ gerundet, und Werte "uber $99.99$ k"onnten gar
nicht erst gespeichert werden. $DECIMAL$ wird h"aufig f"ur finanzielle Daten (Kosten, Preise, etc.) verwendet. 
\subsection{Temporale Daten}
\textbf{DATE} ist der Datentyp f"ur ein Datum ohne Zeitangaben.\\
\textbf{TIME} ist der Datentyp f"ur eine Zeit ohne Datumsangaben.\\
\textbf{DATETIME} ist der Datentyp f"ur eine Kombination der beiden obigen.

Leider gibt es f"ur eine Zeitdauer keinen einheitlichen Datentyp. Eine m"oglichkeit ist es, da"ur \textbf{TIME} zu nehmen. Doch daf"ur muss die
Zeitdauer kleiner als 24 Stunden sein. Anstatt dessen kann man sonst einen Zahlwert nehmen, und die Zahleneinheit am Besten gleich im
Spaltennamen der Tabelle erw"ahnen, damit er nicht vergessen wird. Bei der Tabelle~\ref{tbl:tracks} zum Beispiel kann man die Spalte 
$Duration$ in $DurationSeconds$ umbenennen, und die Werte dann z.B. als \textbf{INT}-Werte speichern (falls einem die Sekundenbruchteile
nicht interessieren).
\subsection{Zeichenketten}
F"ur Zeichenketten gibt es zwei verschiedene Formate:
\textbf{CHAR(l)} ist eine Zeichenkette mit fixer L"ange $l$.\\
\textbf{VARCHAR(l)} ist eine Zeichenkette mit h"ochstl"ange $l$.\\
Wenn man bei einem Feld die H"ochstl"ange nicht kennt, kann man auch f"ur $l$ anstatt eines Zahlenwertes den Ausdruck $max$ nehmen.
Dies kann jedoch bei Textsuchen die Abfrageeffizienz beeintr"achtzigen.
\begin{exercise}
An sich k"onnte man jedes der vorgestellten Datentypen in ein \textbf{VARCHAR(max)} schreiben. Wieso verwendet man dann nicht immer 
\textbf{VARCHAR(max)}? 
\end{exercise}
\begin{exercise}
Bestimme die Datentypen der Beispiele aus dem vorhergehenden Kapitel.
\end{exercise}
\section{Datenbanken erstellen in SQL}
Mit dem Wissen, das wir uns in den letzten Kapitel angeeignet haben, ist es relativ einfach, eine Datenbank mit SQL zu erstellen. Folgendes
Beispiel erstellt die Tabelle Booking von~\ref{tbl:firstExampleEnd}.
\begin{lstlisting}[frame=single, language=SQL, style=SQL]  
CREATE TABLE Booking (
	BookingId INT NOT NULL,
	ServiceFk INT NOT NULL,
	CustomerFk INT,
	StartDate DATETIME NOT NULL,
	CONSTRAINT pk_booking PRIMARY KEY (BookingId),
	CONSTRAINT fk_service FOREIGN KEY (ServiceFk) 
		REFERENCES Service(ServiceId),
	CONSTRAINT fk_customer FOREIGN KEY 
		(CustomerFk) 
		REFERENCES Customer(CustomerKey)
	
);
\end{lstlisting}
Die erste Zeile beschreibt, was wir tun wollen, n"amlich eine Tabelle erstellen. Die folgenden Felder sind die Spalten der Tabelle. Zuerst wird
jeweils der Name der Spalte geschrieben, dann der Datentyp. Wenn eine Spalte nicht leer sein darf, dann wird dies mit NOT NULL
gekennzeichnet. In unserem Beispiel d"urfen auch Buchungen get"atigt werden, bei welchen der Kunde fehlt, da er zum Beispiel nicht 
bekannt ist. Prim"arschl"ussel und Fremdschl"ussel werden mit dem CONSTRAINT-Keyword erstellt. Solche Constraints erhalten einen Namen,
damit man sie sp"ater auch noch modifizieren oder l"oschen kann. Die Namen pk\_booking, fk\_service und fk\_customer stammen aus
Konventionen, wenn man m"ochte, kann man Constraints beliebig nennen. Auf den Namen folgt die Art des Constraints. In dieser Lektion 
werden nur die Contraints ''Primary Key'' und ''Secondary Key'' behandelt. Nach der Art des schl"ussels wird deklariert, auf welche Spalten der
Constraint wirkt. Alle Constraints, welche wir hier eignef"ugt haben, gelten nur f"ur eine einzelne Spalte.
Zeilen 7 bis 12 besch"aftigen sich mit den Fremdschl"ussel. Zus"atzlich zur Information, welche Spalte teil des Schl"ussels ist, wird auch noch
beschrieben, welche Spalten welcher Tabelle referenziert werden.

Hier noch ein Beispiel, wie die Tabelle ~\ref{tbl:tracks}  in SQL aussieht:
\begin{lstlisting}[frame=single, language=SQL, style=SQL]  
CREATE TABLE Track (
	AlbumFk INT NOT NULL,
	TrackId INT NOT NULL,
	Name NVARCHAR(200),
	DurationSeconds INT NOT NULL,
	CONSTRAINT pk_track PRIMARY KEY 
		(AlbumFK, TrackId)
	CONSTRAINT fk_album FOREIGN KEY (AlbumFk)
		REFERENCES Album(AlbumKey)
);
\end{lstlisting}
\begin{exercise}
Erstelle die anderen vier Tabellen dieses Kapitels in SQL.
\end{exercise}
\section{Daten einsetzen}
Eine Datenbank abzufragen macht erst Sinn, wenn Daten drin sind. Dies geschieht mit dem INSERT INTO-Statement. Dieses Statement besteht aus zwei Teilen.
Im ersten Teil wird angegeben, in welche Spalten welcher Tabelle Werte eingesetzt werden. Im zweiten-Teil, nach dem VALUES-Keyword, werden die 
Werte eigesetzt. Das Ganze sieht zum Beispiel f"ur die Booking-Tabelle so aus:
\begin{lstlisting}[frame=single, language=SQL, style=SQL]  
INSERT INTO Booking 
	(BookingKey, ServiceFk, CustomerFk, StartDate)
VALUES (4, 1, 3, '21.11.2014 10:30')
\end{lstlisting}
In diesem Beispiel sehen wir auch, wie wir in SQL mit Datentypen umgehen m"ussen: Zahlenwerte schreiben wir ganz gew"ohnlich aus. Alle
anderen Datentypen, also Zeitdaten und Zeichenketten schreiben wir mit Hochkommas.  
\begin{exercise}
F"uge in jede der 5 anderen Tabellen einen weiteren Datensatz ein. 
\end{exercise}
\chapter{Datenbankabfragen}
\section{Aufbau einer SELECT-Abfragen}
Jetzt dass wir wissen, wie wir unsere Datenbanken erstellen und Daten ein"ugen, k"onnen wir anfangen, Daten abzufragen. Daf"ur verwendet man das SELECT-Keyword. die einfachste Datenabfrage sieht wie folgt aus:
\begin{lstlisting}[frame=single, language=SQL, style=SQL]  
SELECT *
FROM Track
\end{lstlisting}
Eine SQL-Abfrage in der Datenbank gibt immer eine Tabelle zur"uck. Diese Datenbankabfrage gibt uns die ganze Tabelle Track zur"uck. Nach dem SELECT-Keyword muss man eingeben, welche Daten man 
im Resultat erhalten m"ochte. * steht f"ur s"amtliche Spalten. Nach dem FROM steht, von welcher Tabelle man die Felder nehmen will. Wenn
wir zum Beispiel nur die Spalten Name und Duration br"auchten, s"ahe die Abfrage wie folgt aus:
\begin{lstlisting}[frame=single, language=SQL, style=SQL]  
SELECT Name, Duration
FROM Track
\end{lstlisting}
Da der Zweck von SQL ist, Daten "uber mehrere Tabellen abzufragen, ist es oft praktisch, Tabellen in den Abfragen einen neuen Namen zu geben. 
Dann kann es auch n"otig sein, Spalten zusammenzufassen, denn wenn mehrere Tabellen zusammengefasst werden, kann es vorkommen, 
dass gewisse Spalten gleich heissen. Dieses Beispiel macht das gleiche wie das letzte, die Tabelle Track wird jedoch t genannt, und die Track-Spalte
wird zu Trackname umbenannt. 
\begin{lstlisting}[frame=single, language=SQL, style=SQL]  
SELECT t.Name as Trackname, t.Duration
FROM Track t
\end{lstlisting} 
Bei SELECT-Abfragen k"onnen nicht nur Spalten aus der Tabelle, sondern auch Konstanten oder berechnete Werte abgefragt werden. So gibt zum 
Beispiel diese Abfrage eine Tabelle zur"uck, bei welcher alle Eintr"age den Status ''In Library'' haben. Weiter sind in der Tabelle die Spieldauer in
Minuten und der Track-Name in Kleinbuchstaben aufgelistet:
\begin{lstlisting}[frame=single, language=SQL, style=SQL]  
SELECT 'In Library' AS Status, 
	LOWER(t.Name) as Trackname, 
	t.DurationSeconds/60 AS DurationMinutes
FROM Track t
\end{lstlisting} 
Weiter gibt es noch das Schl"usselwort DISTINCT zu erw"ahnen. mit diesem Schl"usselwort werden doppelte Eintr"age eliminiert. Beispielsweise 
gibt uns diese Abfrage zur"uck, wieviel Minuten die Lieder in der Tabelle dauern.
\begin{lstlisting}[frame=single, language=SQL, style=SQL]  
SELECT DISTINCT t.DurationSeconds/60 AS DurationMinutes
FROM Track t
\end{lstlisting}
\section{WHERE-Bedingungen}
\section{ORDER BY}
\section{Unterabfragen}
\section{verschiedene JOINs}

\chapter{Datenbankmanipulationen}
\section{Einf\"{u}gen}
\section{Updates}
\chapter{L"osungen}
\end{document}
