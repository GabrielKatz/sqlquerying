\documentclass[12pt,a4paper]{report}

\setlength{\topmargin}{0 cm}
\usepackage{lmodern}
\usepackage{times}
\usepackage{float}
\usepackage{rotating}
\usepackage[german]{babel}
\usepackage{graphics}
\usepackage{listings}
\usepackage{hyperref}
\usepackage{needspace}
\makeatletter
\newcommand{\lstuppercase}{\uppercase\expandafter{\expandafter\lst@token
                           \expandafter{\the\lst@token}}}
\newcommand{\lstlowercase}{\lowercase\expandafter{\expandafter\lst@token
                           \expandafter{\the\lst@token}}}
\makeatother
\lstdefinestyle{SQL}{basicstyle=\ttfamily,
                        keywordstyle=\lstuppercase,
                        emphstyle=\itshape,
                        showstringspaces=false,
                        numbers=left,
                        }
                        
\graphicspath{ {../img/} }
\usepackage[table,dvipsnames]{xcolor}
\usepackage{caption}
\usepackage{subcaption}
\usepackage{ntheorem}
\usepackage{pifont}
\theoremstyle{plain}
\theorembodyfont{\normalfont}
\theoremprework{\bigskip\needspace{\baselineskip}\hrule}
\theorempostwork{\needspace{\baselineskip}\hrule\bigskip}
\newtheorem{exercise}{Aufgabe}[chapter]
\renewcommand{\labelenumii}{\arabic{enumi}.\arabic{enumii}.}
\newcommand{\algostep}[2]{\noindent\parbox{4cm}{\scalebox{0.5}{\includegraphics{#1}}}
  \hfill
  \parbox{7cm}{#2}
  \vskip -5mm
}
\newcommand{\ucl}{\ding{56}}
\newcommand{\ign}{\rowcolor[gray]{0.5}}
\newcommand{\usd}[1]{{\cellcolor{SkyBlue}#1}}
\newcommand{\sel}[1]{{\cellcolor{LimeGreen}#1}}
\newcommand{\byp}[1]{{\cellcolor{LimeGreen!10}#1}}
\newcommand{\matrixstep}[2]{\noindent\parbox{6cm}{\scriptsize #1}
  \hfill
  \parbox{7cm}{\small #2}
  \vskip 2mm
}

\title{Relationale Datenabfragen mit SQL}
\author{Gabriel Katz}

\begin{document}
\maketitle
\tableofcontents
\chapter{Einleitung}

\section{Worum geht es hier?}
Eines der wichtigsten Zwecke von Computer ist die Speicherung von Daten. Wenn eine Menge von organisierten
Informationen dauerhaft auf einem zentralen Computer gespeichert werden, und diese Daten leicht abgefragt, 
erg\"{a}nzt, bearbeitet und gel\"{o}scht werden kann, ist von einer \textbf{Datenbank} die Rede. So ist zum Beispiel
ein Notizblock (egal ob in Papierform oder in digitaler, wie zum Beispiel Notes oder OneNote) eine stark 
eingeschr"ankte Datenbank. Um relevante Daten zu finden, muss man eventuell Seite f"ur Seite durchgehen. Beim 
Notizblock aus Papier ist auch die "Uberarbeitung der Daten eingeschr"ankt, da man z.B. auf der Seite nur begrenzt
Platz hat. Daher stellt sich die Frage, welche besser zum Bearbeiten geeigneten Wege es gibt, Daten auf dem Computer
zu speichern.

Die relationalen Datenbanken sind die wohl verbreitetste Art von Datenbanken. Was sie genau sind, werden wir im 
n"achsten Abschnitt genauer betrachten. Zwar gibt es zur Zeit einen Trend zu anderen, sogenannten 
NoSQL-Datenbanken, doch auch f"ur diese sind Kenntnisse in relationalen Datenbanken wichtig.
Diese Unterlagen besch"aftigen sich mit SQL, der Standardsprache in relationalen Datenbanken. Den Rest 
dieses Kapitels werden wir mit den relationalen Datenbanken und deren Aufbau besch"aftigen. Kapitel 2 erkl"art was
SQL ist, und zeigt, wie man eine Datenbank in SQL erstellt. Kapitel 3 ist das Kernkapitel dieser Unterlagen und erkl"art
ausf"uehrlich, wie man Daten in SQL abfragt. Schliesslich sehen wir in Kapitel 4, wie man die Anfragetechniken verwendet,
um Daten in SQL zu manipulieren.

Wir arbeiten in dieser Unterrichtseinheit mit 3 verschiedenen Datenbankbeispielen. In den Kapitel selbst verwenden wir zwei
minimale Datenbanken, welche nur 3 Tabellen beinhalten. Die Kapiteltests bearbeiten ein etwas gr"osseres Datenbankmodell.

SQL verf"ugt "uber sehr viele Funktionen, um Berechnungen mit den internen Datentypen durchzuf"uhren. Beispielsweise gibt es eine
Funktionsbibliothek, um Zeitdaten zu vergleichen und zu berechnen. Da diese Unterlagen sich nicht als SQL-Referenz versteht, 
werden solche Funktionen nur dann eingef"uhrt, wenn sie verwendet werden.
\section{Relationale Datenbanken}
Das Konzept f\"{u}r relationale Datenbanken basiert auf einem Paper von Dr. E. F. Codd aus dem Jahre 1970 
namens ``A Relational Model of Data for Large Shared Data Banks''. Dort schlug er vor, Daten in einer Menge von 
\textbf{Tabellen} darzustellen. Dies Tabellen haben einen Namen und mehrere \textbf{Spalten}. Eine Spalte einer 
Tabelle ist durch seinen Namen, seinen Datentyp, und ob sie obligatorisch oder fakultativ ist, definiert. Datentypen 
dr\"{u}cken aus, welche Art von Daten in der Spalte gespeichert werden kann. Beispielsweise hat eine Spalte, in 
welcher ein Datum gespeichert wird einen anderen Datentypen wie eine Spalte mit Zahl oder wie eine Spalte mit Text. 
Wir werden einige Datentypen, welche von SQL unterst"utzt werden im n\"{a}chsten Kapitel etwas mehr im Detail betrachten. \\
Die Daten werden 
dann in den \textbf{Zeilen} der Tabelle gespeichert. Jede Zeile der Tabelle enth\"{a}lt mindestens in allen 
obligatorischen Spalten Daten. Wenn eine Zeile in einer Spalte keine Daten hat, sagt man auch, dass sie NULL enth"alt.
 Normalerweise ist die Anzahl der Spalten der Tabelle (des Datenschemas) 
\"{u}berschaubar, wogegen die Anzahl der Zeilen riesig sein kann. \\
Eine Tabelle in einer relationalen Datenbank hat immer ein Merkmal, an welchem die Zeilen der Tabellen eindeutig
identifiziert werden k"onnen. Dieses Merkmal wird \textbf{Prim\"arschl\"ussel (PK)} genannt. Dieser Prim\"arschl\"ussel 
besteht h\"aufig aus einer einzelnen Tabellenspalte, doch er kann auch aus mehreren Tabellenspalten bestehen.
In diesem Fall spricht man von einem \textbf{zusammengesetzten} Schl\"ussel. Keines der Tabellenspalten des Prim"arschl"ussels 
darf fakultativ sein.  H\"aufig ist der Prim\"arschl\"ussel eine
Spalte, welche nicht eigentliche Daten enth\"alt, sondern nur ein Kennzeichen, zum Beispiel eine Zahl. In den hier gezeigten Beispielen ist der 
Prim\"arschl\"ussel fast immer eine Zahl, doch dies muss nicht immer so sein.\\
Die Kernidee hinter den relationalen Datenbanken, ist, dass diese Tabellen verbunden sind. Doch wie sieht eine 
derartige Verbindung aus? Hier kommen die sogenannten \textbf{Fremdschl\"ussel (FK)} ins Spiel. Ein Fremdschl\"ussel
besteht aus einer oder mehreren Spalten einer Tabelle, welche den Prim\"arschl\"ussel einer anderen Tabelle 
referenzieren. Da zusammengesetzte Fremdschl\"ussel in der Praxis selten auftreten, werden wir diese in den Beispielen
nicht behandeln.\\
\begin{table}
    \begin{tabular}{|llll|}
    \hline
    CustomerId (PK) & FirstName & LastName & Email               \\ \hline
    1               & Hans      & Muster   & hmuster@example.com \\
    2               & Anja      & Tester   & atester@example.com \\
    3               & Ferdinand & Meier    & fmeier@example.com  \\ \hline
    \end{tabular}

    \caption{Customer}
    \label{tbl:firstExampleStart}
	\begin{tabular}{|lll|}
    \hline
    ServiceId (PK) & Name                      & Duration \\ \hline
    1              & Waschen, Schneiden        & 30:00    \\
    2              & Waschen, Schneiden, F"onen & 45:00    \\
    3              & T\"onen                     & 90:00    \\ \hline
    \end{tabular} 
    \caption{Service}
    \begin{tabular}{|llll|}
    \hline
    BookingId &ServiceFk  & CustomerFk  & StartDate        \\ 
       (PK) &(FK(Service.ServiceId)) & (FK(Customer.CustomerId)) &  \\ \hline
    1 &1                  & 3                   & 21.11.2014 10:30 \\
    2 &1                  & 1                   & 24.11.2014 08:30 \\
    3 & 3                  & 3                   & 23.11.2014 15:00 \\ \hline
    \end{tabular}
    \caption {Booking}
    \label{tbl:firstExampleEnd}
\end{table}

Tabellen~\ref{tbl:firstExampleStart} bis~\ref{tbl:firstExampleEnd} zeigen ein Beispiel f\"ur eine einfache Datenbank, wie 
sie in einem Buchungssystem eines Friseursalons vorstellbar ist. In der Tabelle Customer sind die Kundeninformation 
aufgelistet. \"Andert sich beispielsweise die Emailadresse eines Kunden, muss diese lediglich an einer Stelle, n\"amlich 
in der entsprechenden Spalte der Customer-Tabelle angepasst werden. Wir sehen hier, dass der Prim\"arschl\"ussel, 
das Feld CustomerId, keine eigentlichen Information zum Benutzer enth\"alt. Doch wieso wurde dieses Feld 
eingef\"uhrt? Betrachten wir die Alternativen f\"ur den Prim\"arschl\"ussel: Wenn die Spalte Email Prim\"arschl\"ussel
w\"are, entst\"anden zwei Nachteile. Einerseits k\"onnten die Kunden ihre Mailadresse nicht mehr wechseln, da sonst
das eindeutige Merkmal der Kundenzeile verloren gehen w\"urde. Andererseits k\"onnen keine zwei Benutzer die gleiche
Mailadresse teilen. Die Spalten FirstName und LastName sind nicht eindeutig, und k"onnen somit nicht als 
Prim\"arschl\"ussel (auch zusammengenommen nicht) gew\"ahlt werden. Es g\"abe noch die M"oglichkeit, die 
Kombination FirstName, LastName und Email als Prim\"arschl\"ussel zu w\"ahlen, doch wieder g"abe es das Problem,
dass man die Daten nicht mehr "andern darf. Daher verzichtet man auf komplexe Prim\"arschl\"ussel und f\"uhrt lieber eine 
weitere Zeile ein.

Bei der Tabelle Service verh\"alt es sich gleich wie bei der Customer-Tabelle: Da alle anderen Spalten sich \"andern 
k\"onnen, und eventuell nicht eindeutig sind, wurde eine zus\"atzliche Spalte f\"ur den Prim\"arschl\"ussel eingef\"uhrt,

Die Booking-Tabelle hat zwei Fremdschl\"ussel: Die Spalte ServiceFk referenziert die Spalte ServiceId der Tabelle 
Service, und die Spalte CustomerFk referenziert die Spalte CustomerId der Tabelle Customer. Der erste Eintrag
in der Booking-Tabelle bedeutet also, dass Ferdinand Meier (der Kunde mit der CustomerId 3) am 21. November 2014
um 10:30 einen 45-minutigen Termin zum Haare waschen und schneiden hat (der Service mit der ServiceId 1).

In diesem 
Beispiel enden Spalten, welche andere Tabellen referenzieren auf -Fk, und Prim\"arschl\"ussel auf -Id. Es handelt 
sich hierbei um eine \textbf{Namenskonvention}, an welche in allen Beispielen dieser Unterlagen verwendet wird.
 
 \begin{exercise}\label{ex:changeFK}
Was bedeutet es, wenn beim Eintrag mit der BookingId 2 der ServiceFK von 1 auf 2 ge\"andert wird?
\end{exercise}
 \begin{exercise}\label{ex:noComposedKey}
Wieso sollten die Spalten ServiceId und CustomerId keinen zusammengesetzten Prim\"arschl\"ussel f\"ur die 
Tabelle Booking bilden?
\end{exercise}
\begin{table}
    \begin{tabular}{|llll|}
    \hline
    AlbumId (PK) & ArtistName & AlbumName  &   GenreFk  (FK)        \\ \hline
    1               & Red Hot Chilli Peppers      & Californication & 1 \\
    2               & The Ramones      & Rocket To Russia  & 2\\
    3               & The Beastie Boys & Ill Communication & 3\\ \hline
    \end{tabular}

    \caption{Album}
    \label{tbl:secondExampleStart}
	\begin{tabular}{|llll|}
    \hline
    AlbumFk (PK, FK) & TrackId   (PK)  & Name & Duration \\ \hline
    3		& 1 & Sure Shot       & 3:20    \\
    1		& 1 & Around the World       & 3:59    \\
    2       	& 2 & Rockaway Beach  &  2:07    \\
    1         	& 4 & Otherside        & 4:15    \\ \hline
    \end{tabular} 
    \caption{Track}
    \label{tbl:tracks}
    \begin{tabular}{|lll|}
    \hline
    GenreId(PK) &ParentGenreFk (FK) & Name        \\ \hline
    1 &NULL                  & Rock \\
    2 &1                  & Punk Rock \\
    3 & NULL  & Hip-Hop\\ \hline
    \end{tabular}
    \caption {Genre}
    \label{tbl:secondExampleEnd}
\end{table}

Wir betrachten noch ein Beispiel f"ur eine Datenbank. Tabellen~\ref{tbl:secondExampleStart} 
bis~\ref{tbl:secondExampleEnd}  sind Teil einer Musikbibliotheksdatenbank. In diesem Beispiel kommt in der Tabelle Track
ein zusammengesetzter Prim"arschl"ussel vor. Da der Track eines Albums durch das Album und der Tracknummer eindeutig 
gegeben ist, und die Tracknummer und das Album sich nie mehr "andert, eignet sich diese Kombination hervorragend
als zumsammengesetzten Prim"arschl"ussel.

Ein weiteres h"aufiges Muster, dass bei relationalen Datenbanken vorkommt, ist bei der Genre-Tabelle zu finden. Wir sehen,
dass die ParentGenreFk die eigene Tabelle referenziert. Im Beispiel ist Punk Rock ein Subgenre von Rock. Wir stellen 
ausserdem fest, dass gewisse Genres einen NULL-eintrag beim Feld ParentGenreFk haben. Dies ist die Notation, welche
besagt, dass dieses Feld f"ur einen Eintrag nicht verwendet wird. Eine Zeile kann nur in als fakultativ deklarierten Spalten NULL-eintr"age
haben.
 \begin{exercise}\label{ex:NULL}
K"onnte Eintr"age in der Spalte AlbumFk der Tabelle Track leer sein? Wieso bzw. wieso nicht?
\end{exercise}

Die zwei Datenbanken, welche wir in diesem Kapitel betrachteten, werden uns im auch in den n"achsten Kapitel besch"aftigen, und
alle "Ubungen bis auf die Schlusspr"ufung wird sich mit diesen Beispielen befassen. 
Daher empfiehlt es sich, diese Tabellen beim Weiterlesen immer zur Seite zu haben.
\section{Kapiteltest}
In diesem Kapiteltest wird ein etwas Datenmodell eingef"uhrt, welches uns im Rest dieses Kapitels begleiten wird. In dieser Datenbank 
werden Biere, deren Angebot in Bars, und deren Bestellungen von diversen Personen verwaltet. Tabellen~\ref{tbl:testStart} 
bis~\ref{tbl:testEnd} zeigt die Daten dieses Systems. In der Tabelle Bar sind die Bars aufgelistet. Die Brauerei Tabelle listet die Brauereien
im System, und die Kunden erscheinen in der Kunden Tabelle. Wenn eine Bar ein Bier im Sortiment hat, existiert ein entsprechenden Eintrag in
der Sortiment-Tabelle mit dem entsprechenden Verkaufspreis. Da die Preisinformation in dieser Tabelle enthalten ist, k"onnen Biere nicht einfach aus
dem Sortiment gel"oscht werden, da sonst Informationen zu alten Bestellungen verloren gehen w"urden. Anstatt dessen wird der Wert in der 
Deleted-Spalte auf 1 gesetzt. Mehr dazu im Abschnitt~\ref{sec:refint}.
Eine Bestellung ist ein Eintrag in der Bestellungstabelle. Da jedoch die Bestellung mehrere 
verschiedene Biere beinhalten kann, sind diese in einer separaten Tabelle vermerkt. Jede Bestellung kann mehrere verschiedene Bestellte Biere haben.
Diese Tabelle verf"ugt "uber eine Anzahl bestellter Biere, damit nicht jedes einzelne Bier separat aufgelistet werden muss.

\begin{table}
    \begin{tabular}{|lll|}
    \hline
    BarId (PK) & Name & Adresse               \\ \hline
    1               & Vollmond Taverne & Zentralstrasse 53\\
    2               & Si o No  & Ankerstrasse 6\\
    3               & Brasserie Federal   & Bahnhofplatz 15\\ \hline
    \end{tabular}

    \caption{Bar}
    \label{tbl:testStart}
    
\begin{tabular}{|lll|}
    \hline
    BrauereiId (PK) & Name                      & Ort \\ \hline
    1              & Appenzeller        & Appenzell    \\
    2              & W"adi-Br"au & W"adenswil    \\
    3              &Heineken & Amsterdam    \\
    4              & Turbinenbr"au & Z"urich   \\ \hline
    \end{tabular} 
    \caption{Brauerei}
    \begin{tabular}{|lllll|}
    \hline
    BierId &BrauereiFk &Name & Deziliter  & Einkaufpreis        \\ 
       (PK) &(FK(Brauerei.BrauereiId)) &  & & \\ \hline
    1 &1                  & Qu"ollfrisch                   & 5 & 1.30 \\
    2 &1                  & Vollmond                   & 3.3 & 1.30 \\
    3 & 1                  & Brandl"oscher     & 3.3& 1.20 \\ 
    4 &2                  & Blond Premium & 3.3 & 1.50 \\
    5 &2                  & Dunkel                   & 3.3 & 1.60 \\
    6 & 3                  & Lager                   & 5 & 0.90 \\ 
    7 &4                  & Sprint                   & 3.3 & 1.80 \\
    8 &4                  & Start                   &  3.3 & 1.80 \\
    9 & 4                  & Rekord                   &  3.3 & 1.80 \\ \hline
    \end{tabular}
    \caption {Bier}
        \begin{tabular}{|lll|}
    \hline
    KundenId &Name & LieblingsbierFk        \\ 
       (PK) & & (FK(Bier.BierId))  \\ \hline
    1 &Lisa                  &6 \\
    2 &Urs                  & NULL \\
    3 & Heinz                  & 3 \\ 
    4 & Andrea                 & NULL \\
    \hline
    \end{tabular}
    \caption {Kunde}
    \end{table} 
    \begin{table}
     \begin{tabular}{|lllll|}
    \hline
    SortimentId & BarFk &BierFk & Preis  & Deleted     \\ 
    PK& (FK(Bar.BarId) &(FK(Bier.BierId)) & & \\ \hline
    1 &1 &2                  & 4.00 &0\\
    2 &1 &1                  & 6.00 &0\\
    3 & 1 & 3                  & 5.00 &0\\ 
    4 & 1 &7                  & 5.00 &0\\
    5 & 2 &4                  & 5.00 &0\\
    6 & 2 & 7                  & 6.00 &0\\ 
    7 & 2 & 8                  & 6.00 &0\\
    8 & 2 & 9                  & 6.00 &0\\
    9 & 3 & 6                  & 7.00 &0\\
    10 & 3 & 5                  & 5.00 &0\\
    11 & 3 & 4                  & 5.00 &0\\ 
    12 & 3 & 1                  & 7.50 &0\\
    13 & 3 & 2                  & 5.50 &0\\
    14 & 3 & 7                  & 6.00 &0\\
    15 & 3 & 8                  & 6.00 &0\\
    
     \hline
    \end{tabular}
    \caption{Sortiment}
         \begin{tabular}{|llll|}
    \hline
    BestellungId &KundenFk & BarFk & Zeit        \\ 
       (PK) &(FK(Kunde.KundenId)) & (FK(Bar.BarId)) &\\ \hline
    1 &2                  & 1 & 31.12.2014 22:00\\
    2 &1                  & 1 & 31.12.2014 23:23\\
    3 & 1                  & 1 & 31.12.2014 23:24\\ 
    4 &3                  & 2 & 31.12.2014 23:52\\
    5 &4                  & 3 & 1.1.2015 01:21\\
 
    
     \hline
    \end{tabular}
    \caption {Bestellung}
         \begin{tabular}{|lll|}
    \hline
    BestellungFk &SortimentFk & Anzahl        \\ 
       (PK, FK(Bestellung.& (PK, FK(Sortiment. & \\ 
              BestellungId)) &SortimentId)) & \\ \hline
    1&3&2\\
    2&2&1\\
    2&3&2\\ 
    3&4&1\\
    4 &6&3\\
    4&5&1\\ 
    5&12&3\\
   5&9&1\\ 
    
     \hline
    \end{tabular}
    \caption{BestellteBier}
    \label{tbl:testEnd}
\end{table} 
\begin{exercise}
Wir versuchen am Anfang, die Daten zu verstehen. Beschreibe in chronologischer Reihenfolge, welcher Kunde wann wo was bestellt hat.
\end{exercise}
\begin{exercise}
Zu wie viel mal dem Einkaufspreis wird das Vollmond-Bier in der Vollmond-Taverne ausgeschenkt?
\end{exercise}
\begin{exercise}
Wie "andert sich die Datenbank, wenn Urs am 1. Januar 2015 um 2 Uhr im Federal noch zwei Sprint bestellt?
\end{exercise}
\begin{exercise}
Die Tabelle Sortiment hat einen einfachen Prim"arschl"ussel. Was w"urde passieren, wenn wir hier anstatt dessen
einen zusammengesetzten Schl"ussel aus BarFk und BierFk nehmen w"urden.
\end{exercise}
\begin{exercise}
K"onnte man in der Tabelle BestellteBier f"ur den Prim"arschl"ussel eine Id-Spalte einf"uhren? 
\end{exercise}

\chapter{Relationale Datenbanken in SQL}
In diesem Kapitel lernen wir endlich SQL etwas n"aher kennen. Ziel des Kapitels ist es, dass wir die Beispieltabellen in SQL erstellen
k"onnen. Daf"ur untersuchen wir zuerst, was SQL genau ist, dann lernen wir die wichtigsten Datentypen von SQL kennen.
Anschliessend sind wir bereit, das Datenmodell in SQL zu erstellen, in welches wir dan schliesslich die Daten einf"ugen.
\section{Was ist SQL?}
Dr. E. F. Codd, der wie im letzten Kapitel beschrieben die relationale Datenbanken erfunden hat, hat auch eine Sprache 
namens DSL/Alpha entwickelt, um Daten in relationellen Tabellen zu bearbeiten. Nach einigen Weiterentwicklungen durch 
Donald D. Chamberlin und Raymond F. Boyce bei IBM Research entstand dann die Sprache SEQUEL (kurz f"ur Standard 
English Query Language), welche dann aus markenrechtlichen Gr"unden in SQL umbenannt werden musste. SQL ist seit 
1986 von ANSI (American National Standard Institute) standardisiert, und dieser Standard wurde regelm"assig aktualisiert, 
so dass heutzutage auch zum Beispiel die Integration und Abfrage von XML in SQL-Datenbanken standardisiert ist.

Im Gegensatz zu C, C\#, Javascript oder Java ist SQL keine Allzwecksprache. Sie dient lediglich der Bearbeitung und Abfrage
von Relationen Daten. Mit SQL werden lediglich notwendige Ein- und Ausgaben beschrieben. Wie die Daten gespeichert, bzw.
wiedergegeben werden, ist Aufgabe der Datenbankengine. Diese optimiert die Ausf"uhrung der Datenbankabfrage, so dass 
die Daten m"oglichst effizient abgefragt werden.

Eine Software, welche sich umf"anglich mit der Bewirtschaftung von relationalen Datenbanken befasst, nennt sich RDBMS, kurz
f"ur Relational Database Management System. Alle g"angigen RDBMS verwenden SQL oder eine Erweiterung von SQL. Die 
g"angigsten RDBMS-Systeme sind:

\begin{itemize}
\item MySQL, ein Open-Source-System, welches eher wenig Erweiterungen zu SQL verwendet
\item SQL Server von Microsoft. Die SQL-Erweiterung, welche hier verwendet wird, nennt sich T-SQL.
\item Oracle von der gleichnamigen Firma. Die Erweiterte Sprache, welche in Oracle verwendet wird, heisst PL/SQL.
\item SQLite, ein sehr schlankes Open-Source-DBMS. SQLite ist so leicht, dass einige Webbrowser (Safari, Chrome und Opera)
"uber ein eingebautes SQLite verf"ugen.
\item PostgreSQL
\end{itemize}

SQL an sich ist keine vollst"andige Programmiersprache. Viele erweiterte Versionen von SQL, wie PL/SQL,
MySQL oder T-SQL sind jedoch zumindest theoretisch vollst"andige Programmiersprachen. (F"ur die Informatikexperten:
SQL alleine ist nicht Turing-komplett, MySQL, T-SQL, etc. hingegen schon.) Es ist jedoch nicht praktikabel, Datenbankunabh"angige
Anwendungen mit diesen Erweiterungen zu entwickeln. Viele Datenbankanwendungen werden aber nicht direkt mit SQL 
entwickelt, sondern mit einem Toolkit oder einer API aus einer 
Allzwecksprache heraus. Beispiele daf"ur sind JDBC f"ur Java oder das Entity Framework f"ur C\# und Visual Basic.

Man kann SQL ohne irgendetwas zu installieren auf dem Web ausprobieren. Auf 
\url{http://www.w3schools.com/sql/trysql.asp?filename=trysql_select_all} kann man die Webbrowser-interne Datenbankengine
ansteuern. Um die Daten bearbeiten zu k"onnen, muss die Seite jedoch mit Chrome, Opera oder Safari aufgerufen werden,
da die anderen Browser Web-SQL nicht unterst"utzen. Ansonsten kann man auf \url{http://sqlfiddle.com/} mit vielen
verschiedenen Datenbankengines herumexperimentieren.
\section{Datentypen}
In SQL gibt es, "ahnlich wie in Programmiersprachen, verschiedene Datentypen. Es gibt eine Vielzahl von komplexeren Datentypen,
welche f"ur spezifische Zwecke, wie zum Beispiel f"ur Bilder und XML-Dateien existieren. Doch da wir uns vor allem f"ur die
Datenabfrage interessieren, richten wir unseren Fokus auf drei verschiedene Arten von Datentypen: Zahlwerte, temporale Daten und
Zeichenketten.  

\subsection{Zahlwerte}
Ein \textbf{BIT} kann nur entweder 0 (false) oder 1 (true) sein. Dieser Datentyp ist ideal f"ur Antworten auf Ja/Nein-Fragen, welche nicht
nur in der Informatik h"aufig auftreten.
Ein \textbf{INT} ist eine Ganzzahl. Es gibt verschiedene Variationen davon f"ur verschiedene gr"ossen (\textbf{SMALLINT}, \textbf{BIGINT}, etc.),
doch das Grundprinzip ist das Gleiche. In den vorhergehenden Beispielen wurden $INT$s vor allem f"ur Schl"ussel verwendet.
Ein \textbf{FLOAT} ist eine Gleitkommazahl. Dies heisst, dass diese Zahl fast beliebig gross oder klein sein kann, und Stellen nach dem Komma
aufweisen darf.
Ein \textbf{DECIMAL(p,s)} ist eine Zahl mit $p$ Ziffern insgesamt, davon $s$ nach dem Komma. Dies heisst, dass z.B. $41.46$ als 
\textbf{DECIMAL(4,2)} gespeichert werden kann, jedoch wird $31.8734$ auf $31.87$ gerundet, und Werte "uber $99.99$ k"onnten gar
nicht erst gespeichert werden. $DECIMAL$ wird h"aufig f"ur finanzielle Daten (Kosten, Preise, etc.) verwendet. 
\subsection{Temporale Daten}
\textbf{DATE} ist der Datentyp f"ur ein Datum ohne Zeitangaben.\\
\textbf{TIME} ist der Datentyp f"ur eine Zeit ohne Datumsangaben.\\
\textbf{DATETIME} ist der Datentyp f"ur eine Kombination der beiden obigen.

Leider gibt es f"ur eine Zeitdauer keinen einheitlichen Datentyp. Eine m"oglichkeit ist es, da"ur \textbf{TIME} zu nehmen. Doch daf"ur muss die
Zeitdauer kleiner als 24 Stunden sein. Anstatt dessen kann man sonst einen Zahlwert nehmen, und die Zahleneinheit am Besten gleich im
Spaltennamen der Tabelle erw"ahnen, damit er nicht vergessen wird. Bei der Tabelle~\ref{tbl:tracks} zum Beispiel kann man die Spalte 
$Duration$ in $DurationSeconds$ umbenennen, und die Werte dann z.B. als \textbf{INT}-Werte speichern (falls einem die Sekundenbruchteile
nicht interessieren).
\subsection{Zeichenketten}
F"ur Zeichenketten gibt es zwei verschiedene Formate:
\textbf{CHAR(l)} ist eine Zeichenkette mit fixer L"ange $l$.\\
\textbf{VARCHAR(l)} ist eine Zeichenkette mit h"ochstl"ange $l$.\\
Wenn man bei einem Feld die H"ochstl"ange nicht kennt, kann man auch f"ur $l$ anstatt eines Zahlenwertes den Ausdruck $max$ nehmen.
Dies kann jedoch bei Textsuchen die Abfrageeffizienz beeintr"achtzigen.
\begin{exercise}
An sich k"onnte man jede Zeichenkette in ein \textbf{VARCHAR(max)} schreiben. Wieso verwendet man dann nicht immer 
\textbf{VARCHAR(max)}? 
\end{exercise}
\begin{exercise}
Bestimme die Datentypen des Buchungsdatenbank aus dem vorhergehenden Kapitel.
\end{exercise}
\begin{exercise}
Bestimme die Datentypen des Musikdatenbank aus dem vorhergehenden Kapitel.
\end{exercise}
\section{Datenbanken erstellen in SQL}
Mit dem Wissen, das wir uns in den letzten Kapitel angeeignet haben, ist es relativ einfach, eine Datenbank mit SQL zu erstellen. Folgendes
Beispiel erstellt die Tabelle Booking von~\ref{tbl:firstExampleEnd}.
\begin{lstlisting}[frame=single, language=SQL, style=SQL, caption={Erstellen der Tabelle Booking}]  
CREATE TABLE Booking (
	BookingId INT NOT NULL,
	ServiceFk INT NOT NULL,
	CustomerFk INT,
	StartDate DATETIME NOT NULL,
	CONSTRAINT pk_booking PRIMARY KEY (BookingId),
	CONSTRAINT fk_service FOREIGN KEY (ServiceFk) 
		REFERENCES Service(ServiceId),
	CONSTRAINT fk_customer FOREIGN KEY 
		(CustomerFk) 
		REFERENCES Customer(CustomerKey)
	
);
\end{lstlisting}
Da dies der erste SQL-Code in diesen Unterlagen ist, zuerst einige Bemerkungen: Wie in vielen Programmiersprachen d"urfen bei Leerschl"sagen
beliebig viele weitere Leerschl"age oder Zeilenumbr"uche geschrieben werden. Man k"onnte auch diesen Code auf eine Zeile packen, oder jedes 
Wort dieses codes auf eine separate Zeile legen. In diesem Buch ist der Code mit etwas mehr Zeilenumbr"uchen versehen als sonst in der Praxis
"ublich, damit die Lesbarkeit auf Papier einfacher wird. In den Beispielen werden Zeilen, welche sich auf die Vorzeile beziehen einger"uckt, um eine
"Ubersicht "uber den Code zu schaffen. Dies ist eine g"angige Praxis in allen Programmiersprachen.

Wir sehen in diesem Codebeispiel, dass gewisse Worte in Grossbuchstaben geschrieben sind. Das sind die SQL-Schl"usselw"orter. SQL 
unterscheidet eigentlich gar nicht zwischen Klein-und Grossbuchstaben. SQL ist also \textbf{case-insensitive} (Es gibt jedoch Ausnahmen, wie
zum Beispiel der Textvergleich von Datens"atzen). Es hat sich jedoch eingeb"urgert SQL-Schl"usselw"orter in Grossbuchstaben zu schreiben. 
Dem wird auch in diesen Unterlagen Folge geleistet.

Die erste Zeile beschreibt, was wir tun wollen, n"amlich eine Tabelle erstellen. Die folgenden Felder sind die Spalten der Tabelle. Zuerst wird
jeweils der Name der Spalte geschrieben, dann der Datentyp. Wenn eine Spalte nicht leer sein darf, dann wird dies mit NOT NULL
gekennzeichnet. In unserem Beispiel d"urfen auch Buchungen get"atigt werden, bei welchen der Kunde fehlt, da er zum Beispiel nicht 
bekannt ist. Prim"arschl"ussel und Fremdschl"ussel werden mit dem CONSTRAINT-Keyword erstellt. Solche Constraints erhalten einen Namen,
damit man sie sp"ater auch noch modifizieren oder l"oschen kann. Die Namen pk\_booking, fk\_service und fk\_customer stammen aus
Konventionen, wenn man m"ochte, kann man Constraints beliebig nennen. Auf den Namen folgt die Art des Constraints. In dieser Lektion 
werden nur die Contraints ``Primary Key'' und ``Secondary Key'' behandelt. Nach der Art des schl"ussels wird deklariert, auf welche Spalten der
Constraint wirkt. Alle Constraints, welche wir hier eignef"ugt haben, gelten nur f"ur eine einzelne Spalte.
Zeilen 7 bis 12 besch"aftigen sich mit den Fremdschl"ussel. Zus"atzlich zur Information, welche Spalte teil des Schl"ussels ist, wird auch noch
beschrieben, welche Spalten welcher Tabelle referenziert werden.

Hier noch ein Beispiel, wie die Tabelle ~\ref{tbl:tracks}  in SQL aussieht:
\begin{lstlisting}[frame=single, language=SQL, style=SQL, caption={Erstellen der Tabelle Track}]  
CREATE TABLE Track (
	AlbumFk INT NOT NULL,
	TrackId INT NOT NULL,
	Name NVARCHAR(200),
	DurationSeconds INT NOT NULL,
	CONSTRAINT pk_track PRIMARY KEY 
		(AlbumFk, TrackId)
	CONSTRAINT fk_album FOREIGN KEY (AlbumFk)
		REFERENCES Album(AlbumKey)
);
\end{lstlisting}
\begin{exercise}
Erstelle die zwei anderen Tabellen dieser Datenbank in SQL.
\end{exercise}
\begin{exercise}
Erstelle die restlichen zwei Tabellen der Buchungsdatenbank in SQL.
\end{exercise}
\section{Daten einsetzen}
Eine Datenbank abzufragen macht erst Sinn, wenn Daten drin sind. Dies geschieht mit dem INSERT INTO-Statement. Dieses Statement besteht aus zwei Teilen.
Im ersten Teil wird angegeben, in welche Spalten welcher Tabelle Werte eingesetzt werden. Im zweiten-Teil, nach dem VALUES-Keyword, werden die 
Werte eigesetzt. Das Ganze sieht zum Beispiel f"ur die Booking-Tabelle so aus:
\begin{lstlisting}[frame=single, language=SQL, style=SQL, caption={Einf"ugen eines Datensatzes}]  
INSERT INTO Booking 
	(BookingKey, ServiceFk, CustomerFk, StartDate)
VALUES (4, 1, 3, '2014-11-21 10:30')
\end{lstlisting}
In diesem Beispiel sehen wir auch, wie wir in SQL mit Datentypen umgehen m"ussen: Zahlenwerte schreiben wir ganz gew"ohnlich aus. Alle
anderen Datentypen, also Zeitdaten und Zeichenketten schreiben wir mit Hochkommas. Zeitdaten werden normalerweise im Format 
yyyy-MM-dd HH:mm(:ss), also zuerst Jahr, dann Monat, anschliessend Tag geschrieben.

Wir k"onnen auch gleich mehrere Datens"atze in einem Befehl in eine Tabelle einf"ugen. Daf"ur muss man die verschiedenen Datens"atze mit Komma 
voneinander trennen:

\begin{lstlisting}[frame=single, language=SQL, style=SQL, caption={Einf"ugen mehrerer Datens"atze}]  
INSERT INTO Booking 
	(BookingKey, ServiceFk, CustomerFk, StartDate)
VALUES (4, 1, 3, '2014-11-21 10:30'),
       (5, 2, 2, '2014-11-26 14:45')
\end{lstlisting}
In der Praxis ist es h"aufig der Fall, dass Spalten wie BookingKey automatisch generiert werden. Wie man in der Datenbank spezifiziert, dass ein Feld
automatisch generiert werden soll, variiert von System zu System. Daher gehen wir hier nicht n"aher darauf ein. Wenn eine Spalte freiwillig ist, oder
automatisch generiert wird, dann muss man die Spalte im INSERT-Statement nicht angeben. Wenn wir also wissen, dass der BookingKey automatisch
generiert wird, reicht auch schon folgendes zum Einf"ugen der beiden Datens"atze:


\begin{lstlisting}[frame=single, language=SQL, style=SQL, caption={Einf"ugen mit automatisch generiertem Feld}]  
INSERT INTO Booking 
	(ServiceFk, CustomerFk, StartDate)
VALUES (1, 3, '2014-11-21 10:30'),
       (2, 2, '2014-11-26 14:45')
\end{lstlisting}

Wenn man Daten in SQL-Tabellen einf"ugt, m"ussen die Referenzen existieren. Wenn wir versuchen w"urden, eine Buchung mit CustomerFk 4 einzuf"ugen,
w"urden wir einen Fehler zur"uckerhalten, selbst wenn wir in der n"achten Zeile den Kunden mit der CustomerId 4 erstellen w"urden. Die 
Erstellungsreihenfolge ist also wichtig.  
\begin{exercise}
Erstelle sie die Beispielsdatens"atze der Tracks-Datenbank mittels SQL-Skript. 
\end{exercise}
\section{Kapiteltest}
Wir arbeiten wieder mit der Datenbank aus Tabellen~\ref{tbl:testStart} bis~\ref{tbl:testEnd}.
\begin{exercise}
Erstelle die Datenbanktabellen Bestellung und BestellteBier mittels SQL.
\end{exercise}
\begin{exercise}
Urs hat am 1. Januar 2015 um 2 Uhr im Federal noch zwei Sprint bestellt. Bilde dies mittels Skript in die Datenbank ab.
\end{exercise}

\chapter{Datenbankabfragen}
Jetzt dass wir wissen, wie wir unsere Datenbanken erstellen und Daten ein"ugen, k"onnen wir anfangen, Daten abzufragen. Wir beginnen damit, Abfragen
auf eine einzelne Tabelle abzusetzen. Zuerst betrachten wir den Aufbau einer einfachen Abfrage. Anschliessend lernen wir, wie wir unser 
Suchresultat ordnen k"onnen, und wie wir Daten filtern k"onnen, um uns nur auf relevante Informationen zu beschr"anken. Dann lernen wir zwei Methoden
kennen, um Daten aus mehreren Tabellen miteinander zu verkn"upfen: Unterabfragen und JOINs. Schliesslich lernen wir, wie wir mehrere Tabellenzeilen
in ein Resultatfeld zusammen gruppieren k"onnen.
\section{Aufbau einer SELECT-Abfragen}
 Um Daten abzufragen, 
verwendet man das SELECT-Keyword. Die einfachste Datenabfrage sieht wie folgt aus:
\begin{lstlisting}[frame=single, language=SQL, style=SQL, caption={Einfache SQL-Abfrage}]  
SELECT *
FROM Track
\end{lstlisting}
Eine SQL-Abfrage in der Datenbank gibt immer eine Tabelle zur"uck. Diese Datenbankabfrage gibt uns die ganze Tabelle Track zur"uck. Nach dem SELECT-Keyword muss man eingeben, welche Daten man 
im Resultat erhalten m"ochte. * steht f"ur s"amtliche Spalten. Nach dem FROM steht, von welcher Tabelle man die Felder nehmen will. Wenn
wir zum Beispiel nur die Spalten Name und Duration br"auchten, s"ahe die Abfrage wie folgt aus:
\begin{lstlisting}[frame=single, language=SQL, style=SQL, caption={Spezifische Spalten abfragen}]  
SELECT Name, DurationSeconds
FROM Track
\end{lstlisting}
Da ein wichtiger Zweck von SQL ist, Daten "uber mehrere Tabellen abzufragen, ist es oft praktisch, Tabellen in den Abfragen einen neuen Namen zu geben. 
Dann kann es auch n"otig sein, Spalten zusammenzufassen, denn wenn mehrere Tabellen zusammengefasst werden, kann es vorkommen, 
dass gewisse Spalten gleich heissen. Dieses Beispiel macht das gleiche wie das letzte, die Tabelle Track wird jedoch t genannt, und die Track-Spalte
wird zu Trackname umbenannt. 
\begin{lstlisting}[frame=single, language=SQL, style=SQL, caption={Tabelle und Spalten benennen}]  
SELECT t.Name as Trackname, t.DurationSeconds
FROM Track t
\end{lstlisting} 
Bei SELECT-Abfragen k"onnen nicht nur Spalten aus der Tabelle, sondern auch Konstanten oder berechnete Werte abgefragt werden. So gibt zum 
Beispiel diese Abfrage eine Tabelle zur"uck, bei welcher alle Eintr"age den Status ``In Library'' haben. Weiter sind in der Tabelle die Spieldauer in
Minuten und der Track-Name in Kleinbuchstaben aufgelistet. Um den Track-Namen in Kleinbuchstaben zu erhalten, wird die SQL-Funktion LOWER()
verwendet.
\begin{lstlisting}[frame=single, language=SQL, style=SQL, caption={Berechnete Spalten}]  
SELECT 'In Library' AS Status, 
	LOWER(t.Name) as Trackname, 
	t.DurationSeconds/60 AS DurationMinutes
FROM Track t
\end{lstlisting} 
Weiter gibt es noch das Schl"usselwort DISTINCT zu erw"ahnen. mit diesem Schl"usselwort werden doppelte Eintr"age eliminiert. Beispielsweise 
gibt uns diese Abfrage zur"uck, wieviel Minuten die Lieder in der Tabelle dauern.
\begin{lstlisting}[frame=single, language=SQL, style=SQL, caption={DISTINCT}]  
SELECT DISTINCT t.DurationSeconds/60 AS DurationMinutes
FROM Track t
\end{lstlisting}
\begin{exercise}
\label{ex:inverse}
Finde eine Datenabfrage "uber die bestehenden Tabellen, welche die Werte aus Tabelle~\ref{tbl:exbySolution} zur"uckgibt
\begin{table}
    \begin{tabular}{|ll|}
    \hline
    NameLower & DurationsMilliseconds       \\ \hline
    waschen, schneiden & 1800000 \\
    waschen, schneiden, f"onen & 2700000 \\
    t"onen & 5400000\\ \hline
    \end{tabular}

    \caption{Erw"unschtes Resultat aus Aufgabe~\ref{ex:inverse}}
    \label{tbl:exbySolution}
\end{table}

\end{exercise}

\section{Sortieren mit ORDER BY}
Grunds"atzlich ist das Resultat einer SQL-Abfrage ungeordnet. Zwar werden in Praxis die Resultate in den meisten Datenbank-Engines nach 
Prim"arschl"ussel geordnet, doch davon kann man nicht immer ausgehen. Daher existiert die ORDER BY-Klausel, mit welcher man bestimmen kann, 
nach welchen Spalten das Resultat geordnet wird. Diese Spalten m"ussen dabei gar nicht unbedingt im Resultat vorkommen. Die Datenbankengine
sortiert zuerst nach der ersten Spalte, dann nach der zweiten, etc. Wenn nicht anders angegeben, wird aufsteigend sortiert. Mit dem
DESC-Schl"usselwort nach dem Spaltennamen kann man absteigend sortieren. Das folgende Beispiel liefert das Resultat aus Tabelle~\ref{tbl:orderby}.
\begin{lstlisting}[frame=single, language=SQL, style=SQL, caption={Order By}] 
SELECT *
FROM Track t
ORDER BY AlbumFk DESC, Name
\end{lstlisting} 

\begin{table}[H]
	\begin{tabular}{|llll|}    \hline
    AlbumFk (PK, FK) & TrackId   (PK)  & Name & Duration \\ \hline
    3		& 1 & Sure Shot       & 3:20    \\
    2       	& 2 & Rockaway Beach  &  2:07    \\
    1		& 1 & Around the World       & 3:59    \\
    1         	& 4 & Otherside        & 4:15    \\ \hline
    \end{tabular} 
    \caption{Tracks nach Album und Namen sortiert}
    \label{tbl:orderby}
\end{table}
\begin{exercise}
Sortiere die Buchungen von Buchungen absteigend nach Termin.
\end{exercise}
\begin{exercise}
Sortiere die Alben nach Genre, und dann absteigend nach K"unstlername.
\end{exercise}

\section{WHERE-Bedingungen}
Wie wir bereits erw"ahnt haben, k"onnen Tabellen in SQL riesig werden. Um den "Uberblick zu behalten, m"ochte man meistens nur einen Teil der
Resultate zur"uckbekommen. Um dies zu erreichen, kann man die Daten filtern. Der einfachste Weg, die Daten zu beschr"anken, ist es, nur die obersten
n Datens"atze zu holen. Leider ist diese Klausel nicht standardisiert, und der Syntax ist in allen g"angigen Systemen anders (TOP ins T-SQL, LIMIT 
in MySQL und noch mal anders in Oracle). Daher verzichten wir hier auf eine Einf"uhrung dieser Klausel.

Um gezielt nach Bedingungen zu filtern, braucht man in allen Versionen von SQL die WHERE-Klausel. 
In folgendem Beispiel sucht man nach Lieder, welche l"anger als 4 Minuten dauern:
\begin{lstlisting}[frame=single, language=SQL, style=SQL, caption={Einfacher WHERE-Filter}] 
SELECT *
FROM Track t
WHERE DurationSeconds > 4*60
\end{lstlisting} 
Wenn man mehrere Bedingungen zusammen verbinden m"ochte, tut man dies entweder mit OR oder mit AND. Mit diesen zwei 
Verbindungsm"oglichkeiten, dem NOT-Ausdruck und entsprechender Klammerung kann man beliebig komplexe logische Verbindungen zwischen
Bedingungen bauen. Da Auslagenlogik jedoch nicht Teil dieses Kurses ist, betrachten wir ein ganz einfaches Beispiel, in welchem wir nach
allen Tracks, welche nicht am Anfang des Albums sind, und l"anger als 4 Minuten dauern, suchen:
\begin{lstlisting}[frame=single, language=SQL, style=SQL, caption={Mehrere Bedingungen mit AND}] 
SELECT *
FROM Track t
WHERE DurationSeconds > 4*60 AND TrackId <> 1
\end{lstlisting} 
\begin{exercise}
Frage alle Termine in der Woche vom 24. bis 30. November 2014 ab. Bemerkung: Daten kann man gleich wie Zahlen mit den Symbolen
kleiner als (<) und gr"osser als (>) vergleichen. 
\end{exercise}
\subsection{Textvergleich}
Bei Textfelder gibt es die M"oglichkeit, nach Teilstrings zu filtern. Mit dem Schl"usselwort LIKE kann man nach Anfang, Ende oder innerer Teil filtern. 
Das \%-Zeichen dient dabei als Wildcard. Kommt es nur nach dem Suchtext vor (`Text\%'), so wird der Text nach dem Anfang gefiltert, kommt es nur vor dem 
Suchtext vor, so wird nach dem Ende gefiltert (`\%Text'). Wenn das \%-Zeichen auf beiden Seiten des Suchtexts steht, so darf der Suchtext an einer beliebigen 
Stelle vorkommen. In diesem Beispiel filtern wir nach Lieder, deren Name mit ``Sure'' anfangen 
\begin{lstlisting}[frame=single, language=SQL, style=SQL, caption={Textvergleich mit LIKE}] 
SELECT *
FROM Track t
WHERE Name LIKE 'Sure%'
\end{lstlisting} 
\subsection{Vergleich mit NULL}
Wenn ein Feld leer sein darf, muss man mit dem Filtern aufpassen: Felder, welche NULL sind, k"onnen nicht verglichen werden, und werden deshalb
nicht retourniert. So ist zum Beispiel das Resultat folgender Abfrage leer:
\begin{lstlisting}[frame=single, language=SQL, style=SQL, caption={Vergleich gibt keine NULL-Werte zur"uck}] 
SELECT *
FROM Genre
WHERE ParentGenreFk <> 1
\end{lstlisting}
Wir wollten eigentlich nach allen Genres suchen, welche nicht Subgenres von Rock sind. Doch leider wird das NULL nicht verglichen, bzw. scheitert
der Vergleich. Man ist geneigt, das Problem folgendermassen zu beheben, doch dies n"utzt nichts:
\begin{lstlisting}[frame=single, language=SQL, style=SQL, caption={Wirkungsloser NULL-Vergleich}] 
SELECT *
FROM Genre
WHERE ParentGenreFk <> 1 OR ParentGenreFk = NULL
\end{lstlisting}
Das Problem ist nach wie vor, dass der Vergleich mit NULL nicht funktioniert. Wenn man mit einem NULL-Wert vergleicht, muss man immer IS NULL
verwenden, wie folgt:
\begin{lstlisting}[frame=single, language=SQL, style=SQL, caption={Funktionierender NULL-Vergleich}] 
SELECT *
FROM Genre
WHERE ParentGenreFk <> 1 OR ParentGenreFk IS NULL
\end{lstlisting}
\section{Unterabfragen}

\subsection{Mengenabfragen}
Anhand unserer jetztigen Kenntnisse, w"urden wir nach den Kunden, die entweder Tester, Muster oder Test heissen, wie folgt suchen:
\begin{lstlisting}[frame=single, language=SQL, style=SQL, caption={Vergleich mit mehreren Werten}] 
SELECT *
FROM Customer
WHERE LastName = 'Muster' OR LastName = 'Tester' 
	OR LastName = 'Test'
\end{lstlisting}
Dies funktioniert bei nur drei Vergleichen relativ gut. Mit mehr als ca. 10 Namen wird diese Notation jedoch m"uhsam. Daher kann man hier das Keyword IN
verwenden, um die Abfrage k"urzer und lesbarer zu machen:
\begin{lstlisting}[frame=single, language=SQL, style=SQL, caption={ Mengenabfrage }] 
SELECT *
FROM Customer
WHERE LastName IN ('Muster', 'Tester', 'Test')
\end{lstlisting}
\subsection{Unterabfragen}
Eine einfache Art, Tabellen miteinander zu verbinden, sind Unterabfragen. Unterabfragen sind den Mengenabfragen aus dem
vorhergehenden Abschnitt "ahnlich. Lediglich ist die Vergleichsmenge nicht explizit aufgeschrieben, sondern ebenfalls das Resultat einer anderen
Abfrage. Das folgende Beispiel gibt alle Lieder von Bands, welche mit  ``The'' anfangen, zur"uck. Dies wird zweistufig gemacht: 
In der Unterabfrage werden alle Ids der Alben, 
deren K"unstler mit ``The'' anfangen abgefragt. Danach wird die AlbumFk-Spalte mit diese Ids verglichen:
\begin{lstlisting}[frame=single, language=SQL, style=SQL, caption={Unterabfrage}] 
SELECT *
FROM Tracks
WHERE AlbumFk IN (
	SELECT AlbumId
	FROM Album
	WHERE ArtistName LIKE 'The%'
)
\end{lstlisting} 
\begin{exercise}
Erstelle eine Abfrage, welche alle Buchungen zur"uckgibt, deren Service l"anger als 30 Minuten dauert.
\end{exercise}
\section{Verschiedene JOINs}
\subsection{INNER JOIN}
Bisher haben alle unsere Datenbankabfragen nur Werte aus einer Tabelle zur"uckgegeben. Doch wirklich m"achtig wird SQL erst, wenn wir
Tabellen zusammenf"ugen. Dies geschieht zum Beispiel mit INNER JOIN. Die INNER JOIN-Klausel kommt im FROM-Teil der Abfrage zum 
Einsatz, und f"ugt jeweils zwei Tabellen zusammen. Zu einem JOIN-Befehl geh"ort immer auch ein ON-Befehl, welcher beschreibt, unter welcher
Bedingung zwei Datens"atze zusammengef"ugt werden. 
Zum Beispiel erstellen wir so f"ur den Fris"or eine Ansicht, bei welchem er den Namen des Kunden 
und der Zeitpunkt des Termins sieht:
\begin{lstlisting}[frame=single, language=SQL, style=SQL, caption={Einfacher INNER JOIN}, label={lst:firstjoin}] 
SELECT c.FirstName, c.LastName, b.StartDate
FROM Customer c INNER JOIN Booking b
	ON b.CustomerFK = c.CustomerId
\end{lstlisting}

\begin{table}
    \begin{tabular}{|lll|}
    	\hline
    	FirstName &LastName & StartDate        \\ 
       	Ferdinand	&Meier 	& 21.11.2014 10:30 \\
   	Hans	 &	Muster & 24.11.2014 08:30 \\
    	Ferdinand& Meier & 23.11.2014 15:00 \\ \hline
    \end{tabular}
    \caption{JOIN "uber Customer und Booking}
    \label{tab:firstjoin}
\end{table}
Das Resultat dieser Abfrage ist in Tabelle zu sehen. Dabei k"onnen Eintr"age von gewissen Tabellen mehrfach auftreten. In unserem Beispiel
taucht Ferdinand Meier in zwei Eintr"agen auf. Beim INNER JOIN spielt es, im Gegensatz zu anderen JOINS, welche wir sp"ater sehen werden,
keine Rolle, welche Tabelle links und welche rechts des JOINs steht. So k"onnen wir auch mehrere INNER JOINS aneinanderreihen, um beliebig viele Tabellen zusammenf"ugen. Daf"ur hier ein Beispiel, bei welchem
in der Musikdatenbank s"amtliche Informationen zu den Tracks zusammengetragen werden.
\begin{lstlisting}[frame=single, language=SQL, style=SQL, caption={Mehrere INNER JOINs}] 
SELECT t.Name as TrackName, a.ArtistName, 
	a.AlbumName, g.GenreName
FROM Track t 
	INNER JOIN Album a ON t.AlbumFk = a.AlbumId
	INNER JOIN Genre g ON a.GenreFk = g.GenreId 
\end{lstlisting}
\begin{exercise}
Erweitere Listing~\ref{lst:firstjoin} um den Namen des Services und dem Enddatum. Um das Enddatum zu berechnet, kannst du auf die eingebaute
SQL-Funktion DATEADD(minute, x, StartDate) zur"uckgreifen. So erh"altst Du das Datum, welches x Minuten nach dem Startdatum ist.
\end{exercise}
\begin{exercise}
\label{ex:selfjoin}
Verbinde die Genre-Tabelle mit sich selbst, so dass in der ersten Spalte das Untergenre und in der zweiten Spalte das Genre steht.
\end{exercise}
\subsection{OUTER JOIN}
Wie der aufmerksame Leser vielleicht schon bemerkt hat, sind bei einem JOIN nicht alle Eintr"age der verkn"upften Tabellen vorhanden. Zum 
Beispiel taucht in der  Tabelle~\ref{tab:firstjoin} Anja Tester nicht mehr auf. Es kann sein, dass man s"amtliche Eintr"age einer Tabelle erhalten 
m"ochte. Daf"ur ist der OUTER JOIN. Ein LEFT, bzw. ein RIGHT OUTER JOIN stellt sicher, dass jeder Eintrag der Tabelle links, bzw. rechts
des Schl"usselworts mindestens einmal auftritt. Ein FULL OUTER JOIN stellt sicher, dass jeder Eintrag beider Tabellen vorkommt. Wenn man 
bei der Abfrage zu Tabelle~\ref{tab:firstjoin} das INNER JOIN durch ein LEFT OUTER JOIN ersetzen w"urde, so erh"alte man 
Tabelle~\ref{tab:outerjoin}.
\begin{table}[H]
    \begin{tabular}{|lll|}
    	\hline
    	FirstName &LastName & StartDate        \\ 
       	Ferdinand	&Meier 	& 21.11.2014 10:30 \\
   	Hans	 &	Muster & 24.11.2014 08:30 \\
    	Ferdinand& Meier & 23.11.2014 15:00 \\
    	Anja & Tester & NULL \\ \hline
    \end{tabular}
    \caption{LEFT OUTER JOIN}
    \label{tab:outerjoin}
\end{table}
\begin{exercise}
F"uhre von Hand die vier JOINs (INNER / LEFT OUTER / RIGHT OUTER / FULL OUTER)-Operationen auf die Tabelle Genre und sich selbst (wie in
Aufgabe~\ref{ex:selfjoin}) aus.
\end{exercise}

\subsection{JOIN mit Unterabfragen}
Joins sind nicht nur mit bereits existierenden Tabellen m"oglich. Wir k"onnen JOINs auch mit berechneten Tabellen ausf"uhren. Hier ein Beispiel:
\begin{lstlisting}[frame=single, language=SQL, style=SQL, caption={JOIN mit berechneter Tabelle}]
SELECT * 
FROM Album a INNER JOIN
  (SELECT AlbumFk
    'In Library' AS Status, 
    LOWER(t.Name) as Trackname, 
    t.DurationSeconds/60 AS DurationMinutes
  FROM Track t) t2
  ON t2.AlbumFk = a.AlbumKey
\end{lstlisting}
Wenn man mit vielen Unterabfragen arbeitet, kann man sehr schnell die "Ubersicht "uber die verschiedenen Tabellennamen, bzw. Spaltennamen zu
verlieren. Daher empfiehlt es sich dann, selbsterkl"arende Namen einzuf"uhren.    
\section{Aggregierung mit GROUP BY}
Alle Datenbankabfragen, welche wir bisher abgesetzt haben, lieferten Rohdaten zur"uck. Manchmal reicht es jedoch, Statistiken "uber die Daten zu
sehen. Beispielsweise m"ochte man nur die Anzahl Buchungen pro Service sehen, oder man m"ochte nur die Gesamt"lange der Alben
wissen. Um dies zu tun, gruppiert man die Eintr"age nach gewissen Spalten mit dem GROUP BY-Befehl. Die Datenbankengine fasst dann s"amtliche
Zeilen, welche die gleichen Gruppierungsfelder haben, in eine Zeile zusammen. Dabei werden Spalten, nach welchen nicht gruppiert werden, mit
einer Aggregationsfunktion zusammengetragen. Jede Spalte des Resultats muss bei einer GROUP BY-Abfrage entweder ein Gruppierungsfeld oder
eine Aggregationsfunktion sein. M"ogliche Aggregationsfunktionen sind: 
\begin{itemize}
\item COUNT(*), gibt die Anzahl zusammengefassten Datens"atze zur"uck
\item MAX(field) und MIN(field), retourniert den maximalen, bzw. minimalen Wert einer Spalte
\item AVG(field), retourniert den Durchschnitt einer Spalte
\end{itemize}
So sieht die Abfrage f"ur die Anzahl Buchungen pro Service aus:
\begin{lstlisting}[frame=single, language=SQL, style=SQL, caption={Z"ahlen der Buchungen pro Service}] 
SELECT ServiceFk, COUNT(*)
FROM Booking
GROUP BY ServiceFk
\end{lstlisting}
\begin{exercise}
\label{ex:groupBy1}
Erstelle eine Abfrage, welche in der Tabelle Tracks die AlbumFk und die Gesamtl"ange des Albums auflistet.
\end{exercise}
\subsection{GROUP BY mit JOINS}
Die letzte Abfrage w"are viel sch"oner, wenn wir gleich den Servicenamen dazu nehmen w"urden. Dies k"onnen wir mit JOIN zusammenf"ugen.
Die Abfrage sieht etwa so aus:
\begin{lstlisting}[frame=single, language=SQL, style=SQL, caption={Versuch, GROUP BY mit JOIN zu kombinieren.}] 
SELECT ?, COUNT(*)
FROM Booking b INNER JOIN Service s
	ON s.ServiceKey = b. ServiceFk
GROUP BY b.ServiceFk
\end{lstlisting}
Jetzt stellt sich die Frage, was beim Fragezeichen stehen sollte. Da s.Name noch kein Gruppierungsfeld ist, m"usste es aggregiert werden.
Es gibt hier 3 M"oglichkeiten, weiterzufahren. Alle M"oglichkeiten f"uhren zum Ziel, doch die letzte ist zu bevorzugen. 
\begin{enumerate}
\item Wir verwenden einfach eine Aggregierung, welche etwas Sinnvolles zur"uckliefert, wie zum Beispiel MAX(s.Name). Dies funktioniert,
da Zeichenketten in SQL auch geordnet werden k"onnen, und das Maximum von gleichen Werten den gleichen Wert hat. Dennoch ist diese
L"osung eher ein Hack, der f"ur Nichteingeweihte schwer verst"andlich ist.
\item Wir gruppieren einfach auch nach s.Name. Auch diese L"osung funktioniert einwandfrei. Da alle Eintr"age mit dem gleichen ServiceFk
den gleichen Servicenamen haben, macht diese zus"atzliche Gruppierung nichts Sichtbares. Diese L"osung macht jedoch in den meisten
Datenbankengines die Abfrage ineffizient, und ist auch unsch"on, da nach so wenig Felder wie m"oglich gruppiert werden soll.
\item Wir gruppieren in einer Unterabfrage. In einer Unterabfrage erstellen wir eine Zwischentabelle bookingByService, welche die ServiceFks
und die Anzahl als Spalten hat, dann JOINen wir diese Tabelle mit der Booking Tabelle. Dies sieht zwar auf den ersten Blick komplizierter aus, 
ist jedoch viel deutlicher. Die Abfrage sieht dann so aus:
\begin{lstlisting}[frame=single, language=SQL, style=SQL, caption={JOIN mit gruppierter Tabelle}] 
SELECT s.Name, BookingCount
FROM Service s INNER JOIN 
  (SELECT ServiceFk, COUNT(*) AS BookingCount
  FROM Booking
  GROUP BY ServiceFk) bookingByService
  ON bookingByService.ServiceFk = s.ServiceKey
\end{lstlisting}
Eine kleine Anmerkung: Diese Technik muss nur dann angewendet werden, wenn die Aggregation eigentlich nur "uber eine Tabelle stattfindet.
Es gibt F"alle, in welchen Daten aus mehreren Tabellen zusammengerechnet und dann aggregiert werden. Dann braucht man diese Technik nicht. 
\end{enumerate} 
\begin{exercise}
Erweitere die Abfrage aus Aufgabe~\ref{ex:groupBy1}, so dass der K"unstlername und der Albumname neben der Gesamtspieldauer des Albums steht.
\end{exercise}
\subsection{Filtern bei Gruppierungen}
Bei einer Gruppierungsabfrage kann man wie gewohnt mit einer WHERE-Klausel "uber die Datens"atze filtern. Diese Filterung geschieht dann vor der
Aggregation. Zum Beispiel kann man wie folgt nur die Buchungen ab dem 24. November f"ur die Statistik ber"ucksichtigen:
\begin{lstlisting}[frame=single, language=SQL, style=SQL, caption={Filtern vor Gruppierung}] 
SELECT ServiceFk, COUNT(*)
FROM Booking
WHERE StartDate > '2014-11-24'
GROUP BY ServiceFk
\end{lstlisting}
Es gibt jedoch auch die M"oglichkeit, "uber Aggregationen zu filtern. Dies geschieht mit der HAVING-Klausel. Diese zus"atzliche Filterung steht 
nach der GROUP BY-Klausel. In diesem Beispiel suchen wir alle Services, welche ab dem 24. November genau ein mal gebucht wurden:
\begin{lstlisting}[frame=single, language=SQL, style=SQL, caption={Filtern vor und nach Gruppierung}] 
SELECT ServiceFk, COUNT(*)
FROM Booking
WHERE StartDate > '2014-11-24'
GROUP BY ServiceFk
HAVING COUNT(*) = 1
\end{lstlisting}
Die Aggregation, welche in der HAVING-Klausel erw"ahnt wird, muss nicht zwingend die gleiche sein wie die im Resultat, wie sie in der n"achsten "Ubung
sehen werden
\begin{exercise}
Erstelle eine Abfrage, welche die Gesamtdauer der Alben, bei welchen die Durchschnittsdauer der Tracks weniger als 3 Minuten betr"agt, berechnet.
\end{exercise} 
\begin{exercise}
Gebe bei den folgenden Bedingungen an, ob eine WHERE oder eine HAVING-Klausel gebraucht wird. Die Abfrage muss nicht erstellt werden.
\begin{itemize}
\item Finde die Anzahl Buchungen aller Kunden, welche schon mindestens zwei Buchungen get"atigt haben.
\item Finde die Anzahl Buchungen des ``T"onen''-Services aller Kunden.
\item Finde die Alben, welche mehr als 10 Tracks haben.
\item Finde die durchschnittliche Dauer der ersten f"unf Tracks jedes Albums.
\end{itemize} 
\end{exercise}
\begin{exercise}
Ein Bonustrack in einem Album ein Track, deren TrackId gr"osser ist als die Anzahl Tracks des Albums. In unserem
Beispiel sind Rockaway Beach und Otherside Bonustracks, da in unserem unvollst"andigen Beispiel Album 2 nur ein Track hat, bzw. Album 2 nur
zwei Tracks hat. Erstelle eine Abfrage, welche die Alben-Ids aller Alben mit Bonustracks zur"uckgibt.
\end{exercise}
\section{Kapiteltest}
Wieder arbeiten wir mit der Datenbank aus Tabellen~\ref{tbl:testStart} bis~\ref{tbl:testEnd}.
\begin{exercise}
Erstelle eine Abfrage, welche die vollst"andige Tabelle Bar zur"uckgibt
\end{exercise}
\begin{exercise}
Erstelle eine Abfrage, welche die Tabelle x zur"uckgibt.
\end{exercise}
\begin{exercise}
Welche Flaschengr"ossen gibt es in dieser Datenbank? Erstelle dazu eine Abfrage. 
\end{exercise}
\begin{exercise}
Erstelle eine Abfrage, welche die Tabelle Sortiment nach Bier geordnet zur"uckgibt.
\end{exercise}
\begin{exercise}
Erstelle eine Abfrage, welche alle Biere, die im Namen ein Umlaut ("a, "o oder "u) enthalten, anzeigt.
\end{exercise}
\begin{exercise}
Zeige alle Biere der Brauerei mit der Id 1 mittels Datenbankabfrage an.
\end{exercise}
\begin{exercise}
Zeige alle Biere an, deren Einkaufspreis pro Deziliter unter 40 Rappen ist.
\end{exercise}
\begin{exercise}
Erstelle eine Abfrage, welche den Namen jedes Kunden und den Namen seines Lieblingsbier anzeigt. Dabei sollen auch Kunden angezeigt werden,
welche kein Lieblingsbier haben.
\end{exercise}
\begin{exercise}
Erstelle eine Abfrage, welche den Namen jedes Kunden und den Namen seines Lieblingsbier anzeigt. Dabei sollen auch Kunden angezeigt werden,
welche kein Lieblingsbier haben. Ausserdem sollen auch gleich die Biere, welche von keinem Kunden favorisiert werden, ausgegeben werden
\end{exercise}
\begin{exercise}
Erstelle eine Abfrage, welche f"ur jeden Eintrag in der Tabelle BestellteBier den Kundennamen, den Biernamen und das Datum retourniert.
\end{exercise}
\begin{exercise}
Erstelle eine Abfrage, welche bei jeder Bestellung KundenId, BarId und Gesamtpreis zur"uckgibt.
\end{exercise}
\begin{exercise}
Erstelle eine Abfrage, welche f"ur jede BarId, welche von Bestellungen referenziert wird, den Umsatz f"ur die Silvesternacht 
(31.12.2014 18:00 bis 1.1.2015 06:00) zur"uckgibt.
\end{exercise}
\begin{exercise}
Zeige die Namen der Personen an, welche in der Silvesternacht mehr als 20 Franken f"ur Bier ausgegeben haben.
\end{exercise}
\begin{exercise}
Erstelle eine Abfrage, welche den Biernamen und dessen g"unstigsten Verkaufspreis zur"uckgibt. Tipp: Die Abfrage muss ber"ucksichtigen, ob das
Bier im Sortiment als gel"oscht markiert wurde.
\end{exercise}
\chapter{Datenbankmanipulationen}
Wir haben jetzt gelernt, wie man Datenbanken erstellt, wie man Datens"atze einf"ugt, und wie man diese Daten abfragt. Was jetzt noch fehlt ist das
eigentliche Arbeiten mit den Daten, also wie man Datens"atze bearbeitet und l"oscht. Diese Operationen sind einfacher zu verstehen, wenn man das
Abfragen von Daten bereits beherrscht. Daher wurden sie bis jetzt vorenthalten. In diesem Kapitel sehen wir auch, wie wir anhand bereits vorhandener Daten
weitere Daten generieren kann.
\section{L"oschen}
Die L"oschabfrage ist der einfachen SELECT-Abfrage sehr "ahnlich. Mit diesem Befehl w"urden s"amtliche Datens"atze der Tabelle Booking gel"oscht:
\begin{lstlisting}[frame=single, language=SQL, style=SQL, caption={L"oschen aller Zeilen einer Tabelle}] 
DELETE
FROM Booking
\end{lstlisting} 
Nat"urlich kommt es sehr selten vor, dass man alle Daten aus einer Tabelle l"oschen muss. Daher kommt der WHERE-Filter beim l"oschen h"aufig zum 
Einsatz.   
\begin{lstlisting}[frame=single, language=SQL, style=SQL, caption={L"oschen von gefilterten Zeilen}] 
DELETE
FROM Booking
WHERE BookingId = 3
\end{lstlisting}
\begin{exercise}
L"osche alle Tracks, welche mit einem Leerzeichen enden.
\end{exercise}
\subsection {L"oschen mit JOINs}
Analog zu den SELECT-Abfragen kann es auch bei DELETEs notwendig sein, die Daten "uber mehrere Tabellen herauszufiltern, bevor man sie l"oscht.
In diesem Fall muss man spezifizieren, von welchen Tabellen man die Daten l"oschen m"ochte. In folgendem Beispiel werden s"amtliche Buchungen
von Kunden gel"oscht, die Tester heissen. Die Kunden selbst werden jedoch nicht aus der Datenbank gel"oscht:
\begin{lstlisting}[frame=single, language=SQL, style=SQL, caption={L"oschen mit JOIN}] 
DELETE b
FROM Booking b INNER JOIN Customer c
  ON b.CustomerFk = c.CustomerKey
WHERE c.LastName = 'Tester'
\end{lstlisting}
\begin{exercise}
L"osche alle Tracks vom K"unstler `Various Artists'  
\end{exercise}

\section{Referenzielle Integrit"at}
\label{sec:refint}
Im SQL muss man beim Modifizieren und L"oschen aufpassen. Eine "Anderung in der Datenbank kann nicht vorgenommen werden, wenn
ein Fremdschl"ussel danach ins Leere zeigt. So kann man zum Beispiel in der Tabelle Service nicht pl"otzlich den Service mit der Id 1 l"oschen,
da dieser in der Tabelle Booking bereits referenziert wird. Wenn man einen referenzierten Datensatz l"oschen m"ochte, muss man s"amtliche
referenzierende Datens�tze auch gleich l"oschen. Die Datenbank kann so eingestellt sein, dass sie dies gleich automatisch tut. Diese Einstellung
heisst CASCADE. Wir betrachten die Option hier nicht n"aher.

Manchmal m"ochte man die referenzierende Datens"atze gar nicht l"oschen. Es gibt zwei M"oglichkeiten, wie der Datenbankdesigner dies zulassen
kann. Die erste Option ist das sogenannte SET NULL. Auf diese kann nur zur"uckgegriffen werden, wenn die Referenzspalte leer, bzw. NULL sein
darf. Wird bei dieser Option ein Datensatz gel"oscht, werden alle Referenzen auf NULL gesetzt. Die zweite M"oglichkeit, wie mit gel"oschten
Datens"atzen umgegangen werden kann, ist mit dem sogenannten Deleted-Flag. Dies ist eine zus"atzliche Spalte in der Tabelle, welche meistens
`Deleted' oder "ahnlich heisst und ein BIT
beinhaltet. Wenn man Daten l"oschen m"ochte, wird in diesem Fall der Datensatz nicht aus der Datenbank entfernt. Statt dessen wird die 
Deleted Spalte einfach auf 1 gesetzt. Dies muss dann bei der Datenbankabfrage ber"ucksichtigt werden. Wenn wir bei der Tabelle Service ein
Deleted verwenden w"urden, m"usste man zum Beispiel wie folgt die Anzahl Services z"ahlen:
  \begin{lstlisting}[frame=single, language=SQL, style=SQL, caption={Delete-Flag}] 
SELECT COUNT(*)
FROM Service
WHERE Deleted = 0
\end{lstlisting}
\section{Updates}
Der UPDATE-Befehl funktioniert analog zum SELECT- und zum DELETE-Befehl.  Der einzige Unterschied ist, dass man spezifizieren muss, welche
Felder wie aktualisiert werden m"ussen. In diesem Beispiel verbessern wir endlich einen Schreibfehler in der urspr"unglichen Datenbank:
 \begin{lstlisting}[frame=single, language=SQL, style=SQL, caption={Einfaches Update}] 
UPDATE Album
SET ArtistName = 'Red Hot Chili Peppers'
WHERE AlbumId = 1
\end{lstlisting} 
Man kann auch gleich mehrere Spalten gleichzeitig in der Datenbank bearbeiten. Dazu reiht man die Felder mit Komma getrennt aneinander, wie in 
folgendem Beispiel. Anmerkung: Dieser Syntax funktioniert nicht in PL/SQL.
 \begin{lstlisting}[frame=single, language=SQL, style=SQL, caption={Update mehrerer Spalten}] 
UPDATE Service
SET Name = 'Nur Waschen', 
  DurationMinutes = DurationMinutes - 5
WHERE ServiceId = 1
\end{lstlisting} 
Dieses Skript setzt die Dauer des Services um f"unf Minuten hinunter. Dieser Syntax ist f"ur nichtprogrammierer etwas gew"ohnungsbed"urftig.
Wenn ein Spaltenname auf der rechten Seite des Gleichs steht, repr"asentiert der Spaltenname den Wert dieser Spalte vor des Updates.  
\begin{exercise}
Angenommen, wir erweitern die Tabelle Customer um eine Spalte FullName, welche aus FirstName, LastName und einem Leerzeichen
besteht. Schreibe eine UPDATE-Skript, welches diese Spalte bef"ullt. Tipp: Um Zeichenketten aneinaderzuh"angen, kann man die SQL-
Funktion CONCAT(,) verwenden. Diese Funktion kann mit beliebig vielen Parameter aufgerufen werden. 
\end{exercise}
\subsection{Updates mit JOINs}
Um Daten von anderen Tabellen der Datenbank f"ur ein Update zu verwenden, empfiehlt es sich, JOINS zu verwenden. Leider ist dies nicht im 
SQL-Standard enthaten. Daher variiert der Syntax
dieser sehr n"utzlichen Funktionalit"at sehr stark zwischen den verschiedenen SQL-Erweiterungen. Daher hier ein Beispiel, welches zumindest in
MySQL funktioniert. In den meisten Erweiterungen sieht diese Abfrage "ahnlich (jedoch etwas komplizierter) aus.
In diesem Beispiel nehmen wir an, dass die Track-Tabelle um die Spalte ArtistName erweitert worden ist, und f"ullen nun diese
Spalte mit den Daten aus der Album-Tabelle ab:
\begin{lstlisting}[frame=single, language=SQL, style=SQL, caption={Update mit JOIN in MySQL}] 
UPDATE Album a INNER JOIN Track t 
  ON a.AlbumKey = t.AlbumFk
SET t.ArtistName = a.ArtistName
\end{lstlisting} 
Der Vollst"andigkeit halber zeigen wir hier noch wie das Gleiche im SQL-Standard aussieht. Hier verwendet man eine Unterabfrage, um
den Wert des K"unstlernamen zu erhalten:
 \begin{lstlisting}[frame=single, language=SQL, style=SQL, caption={Standardisierter Update mit JOIN}] 
UPDATE Track t 
SET t.ArtistName = (
  SELECT a.ArtistName
  FROM Album a
  WHERE a.ArtistKey = t.AlbumFk
)
\end{lstlisting} 


\begin{exercise}
Angenommen, wir erweitern die Tabelle Booking um eine Spalte Comment, in welcher Text geschrieben werden kann.
 Schreibe eine UPDATE-Skript, welches diese Spalte mit Vornamen, Nachnamen und Servicenamen bef"ullt.
 Tipp: Wie in der letzten Aufgabe wird hier auch die CONCAT()-Funktion ben"otigt. Dazu d"urfen wir annehmen, dass wir MySQL verwenden. 
 \end{exercise}
 \begin{exercise}
Angenommen, wir erweitern die Tabelle Booking um eine Spalte EndDate, in welcher das Enddatum des Termins geschrieben werden soll.
Schreibe eine UPDATE-Skript, welches diese Spalte f"ullt.
\end{exercise}
\section{Einf\"{u}gen}
Wir k"onnen auch mittels Unterabfrage das Generieren von Daten erleichtern. Als Beispiel nehmen wir hier wieder das Fris"orunternehmen.
Dieses m"ochte nun seine Kunden "uber das Internet Termine buchen lassen, und m"ochte f"ur jedes seiner Kunden ein Login erstellen. Diese Logins
werden in eine Tabelle User gespeichert, welche "uber die Pflichtfelder CustomerFk, Login und IsPasswordSet. Folgendes Skript
erstellt einen Benutzer f"ur jeden Kunden, bei welchem das Passwort noch nicht gesetzt ist, und dessen Login der Emailadresse des Benutzers
entspricht.
 \begin{lstlisting}[frame=single, language=SQL, style=SQL, caption={Daten generieren}] 
INSERT INTO User (CustomerFk, Login, IsPasswordSet)
SELECT (CustomerId, Email, 0) 
FROM Customer
\end{lstlisting} 
\section{Kapiteltest}
Wieder arbeiten wir mit der Datenbank aus Tabellen~\ref{tbl:testStart} bis~\ref{tbl:testEnd}.
\begin{exercise}
Erstelle ein Skript aus der Tabelle BestellteBier alle Zeilen mit Anzahl gr"osser als 1 l"oscht.
\end{exercise}
\begin{exercise}
Erstelle ein Skript, welch die Bestellung 4 mit s"amtlichen Bestellungsdetails aus der Datenbank l"oscht.
\end{exercise}
\begin{exercise}
Wann darf man Datens"atze aus der Tabelle Bier l"oschen?
\end{exercise}
\begin{exercise}
Die Bestellung 5 wurde falsch erfasst. es wurden nicht drei Bier mit SortimentId 12 und ein Bier mit SortimentId 9 bestellt, sondern je zwei Bier.
Passe dies mittels SQL-Skript an.
\end{exercise}
\begin{exercise}
Die Preise der Bar mit der Id 2 waren falsch in der Datenbank: Sie sind eigentlich 50 Rappen teurer. Dies wurde beim Kassieren bereits ber"ucksichtigt.
Erstelle daf"ur ein SQL-Skript.
\end{exercise}
\begin{exercise}
Die Bar mit der Id 2 "andert ihre Preise: Alle Biere werden 50 Rappen teurer. Erstelle daf"ur ein SQL-Skript. Wir d"urfen annehmen, dass die Bar noch
keine gel"oschte Eintr"age im Sortiment hat. Ausserdem wird die Spalte SortimentId automatisch generiert und muss beim Einf"ugen nicht ber"ucksichtigt
werden. 
\end{exercise}
\chapter{L"osungen}
\section{Kapitel 1}
\begin{enumerate}
\item Wenn die ServiceFk sich "andert, wird ein anderer Service referenziert. Der Service 2 dauert 15 Minuten l"anger als der Service 1 und dauert daf"ur
15 Minuten l"anger.
\item Der Prim"arschl"ussel ist ein eindeutiges Merkmal einer Zeile. Wenn wir ServiceId und CustomerId als zusammengesetzten Prim"arschl"ussel 
verwenden w"urden, d"urfte f"ur ein Kunde ein Service nur genau ein Mal im System erfasst sein. Dies heisst also, dass ein Kunde nicht mehrere Termine
f"ur einen Service gleichzeitig haben kann. Sogar wenn dies akzeptabel w"are, gibt es jedoch einen anderen Grund, welcher dagegen spricht: Beim n"achsten
Termin des gleichen Typs m"usste man in der Datenbank die alte Buchung "uberschreiben. Somit w"aren die historischen Daten unbrauchbar.  
\item Obwohl es in einer Musiksammlung vorkommen kann, dass das Album bei gewissen Tracks nicht bekannt sind, wird dies in dieser Datenbank keine
Rechnung getragen. Die Spalte AlbumFk ist Teil des Prim"arschl"ussels, und darf somit nicht fakultativ sein, also keine NULL-eintr"age enthalten.
\item 
\begin{itemize}
\item Am 31.12. um 22 Uhr hat Urs in der Vollmond Taverne 2 Brandl"oscher bestellt.
\item Um 23 Uhr 23 hat Lisa auch in der Vollmond Taverne 2 Brand"oscher und ein Qu"ollfrisch bestellt.
\item Gleich danach hat Lisa am gleichen Ort noch ein Sprint bestellt.
\item Um 23 Uhr 52 hat Heinz im Si o No 3 Sprint und ein W"adi-Br"au Blond Premium bestellt.
\item Am Neujahrstag um 1 Uhr 21 hat Andrea in der Brasserie Federal 3 Qu"ollfrisch und ein Heineken Lager bestellt.
\end{itemize}
\item Das Vollmond-Bier kostet im Einkauf 1.30 Franken, und wird in der Vollmond Taverne zu 4 Franken ausgeschenkt. Also wird das Bier zu etwas mehr als
3 mal dem Einkaufspreis verkauft.
\item In die Tabellen Bestellung und Bestellte Bier werden folgende Zeilen eingef"ugt:
\begin{table}[H]
         \begin{tabular}{|llll|}
    \hline
    BestellungId &KundenId & BarId & Zeit        \\ 
       (PK) &(FK(Kunde.KundenId)) & (FK(Bar.BarId)) &\\ \hline
    6 &2                  & 3 & 1.1.2015 02:00\\
   
    
     \hline
    \end{tabular}
    \caption {Bestellung}
         \begin{tabular}{|lll|}
    \hline
    BestellungFk &SortimentFk & Anzahl        \\ 
       (PK, FK(Bestellung.& (PK, FK(Sortiment. & \\ 
              BestellungId)) &SortimentId)) & \\ \hline
    6&14&2\\
    
     \hline
    \end{tabular}
    \caption{BestellteBier}
\end{table}
\item Wenn der Schl"ussel zusammengesetzt w"are, w"urde eine Bar pro Bier genau einen Preis haben. Auf den ersten Blick macht das Sinn, jedoch nicht
in Kombination mit der Delete-Spalte. Die Delete-Spalte erm"oglicht es uns, den Preis jedes bestellten Bieres nachzuvollziehen. Diese M"oglichkeit geht jedoch
verloren, wenn wir nur einen Preis pro Bier und Bar speichern k"onnen.
\item Eine Spalte BestellteBierId k"onnte man einf"uhren und als alleiniger Prim"arschl"ussel w"ahlen. Die Datenbank w"urde jedoch eine sinnvolle
Einschi"ankung verlieren. Pro Bestellung k"onnte es n"amlich mehrere Eintr"age mit dem gleichen Bier, bzw. dem gleichen Eintrag im Sortiment geben. Das
kann gewisse Berechnungen etwas komplexer machen, w"urde jedoch der Funktionalit"at der Datenbank nicht schaden. 
\end{enumerate}
\section{Kapitel 2}
\begin{enumerate}
\item Es macht durchaus Sinn, wenn man die L"ange einer Zeichenkette einschr"ankt. Einerseits, weil andere L"angen keinen Sinn ergeben. Beispielsweise
haben die Kantonsk"urzel (z.b. ZH, ZG, SG, etc.) immer genau zwei Buchstaben. Auch gibt es weltweit keine Ortsnamen, welche l"anger als 150 Zeichen lang sind.
Andererseits kann man somit auf der Datenbank die L"ange gewisser Textgr"ossen einschr"anken. Nicht jeder Text soll beliebig lange sein d"urfen.
\item In dieser L"osung verzichten wir darauf, Zeichenketten zu beschr"anken. Auch sind andere Formate f"ur die Zeitdauerwerte vorstellbar.
\begin{itemize}
\item Customer:\\
	CustomerId INT \\
	FirstName VARCHAR(MAX)\\
	LastName VARCHAR(MAX)\\
	Email VARCHAR(MAX)
\item Service:\\
	ServiceId INT \\
	Name VARCHAR(MAX)\\
	DurationMinutes INT
\item Booking:\\
	BookingId INT \\
	ServiceFk INT\\
	CustomerFk INT\\
	StartDate DATETIME	
\end{itemize} 
\item \begin{itemize}
\item Album:\\
	AlbumId INT \\
	ArtistName VARCHAR(MAX)\\
	AlbumName VARCHAR(MAX)\\
	GenreFk INT
\item Track:\\
	AlbumFK INT \\
	TrackId INT\\
	Name VARCHAR(MAX)\\
	DurationSeconds INT
\item Genre:\\
	GenreId INT \\
	ParentGenreFk INT\\
	Name VARCHAR(MAX)	
\end{itemize}
\item Bei vielen Zeilen steht es uns frei, ob sie NULL sein d"urfen oder nicht.  
\begin{lstlisting}[frame=single, language=SQL, style=SQL, caption={Erstellen der Tabelle Album}]  
CREATE TABLE Album (
	AlbumId INT NOT NULL,
	ArtistName VARCHAR(200),
	AlbumName VARCHAR(200),
	GenreFk INT,
	CONSTRAINT pk_album PRIMARY KEY 
		(AlbumId),
	CONSTRAINT fk_genre FOREIGN KEY (GenreFk)
		REFERENCES Genre(GenreId)
);
\end{lstlisting}
\begin{lstlisting}[frame=single, language=SQL, style=SQL, caption={Erstellen der Tabelle Genre}]  
CREATE TABLE Genre (
	GenreId INT NOT NULL,
	ParentGenreFk INT,
	Name VARCHAR(MAX),
	CONSTRAINT pk_genre PRIMARY KEY 
		(GenreId),
	CONSTRAINT fk_genre FOREIGN KEY (ParentGenreFk)
		REFERENCES Genre(GenreId)
);
\end{lstlisting}
\item Bei vielen Zeilen steht es uns frei, ob sie NULL sein d"urfen oder nicht.
\begin{lstlisting}[frame=single, language=SQL, style=SQL, caption={Erstellen der Tabelle Album}]  
CREATE TABLE Customer (
	CustomerId INT NOT NULL,
	FirstName VARCHAR(200),
	LastName VARCHAR(200),
	Email VARCHAR(MAX),
	CONSTRAINT pk_customer PRIMARY KEY 
		(CustomerId)
		
);
\end{lstlisting}
\begin{lstlisting}[frame=single, language=SQL, style=SQL, caption={Erstellen der Tabelle Genre}]  
CREATE TABLE Service (
	ServiceId INT NOT NULL,
	Name VARCHAR(MAX),
	DurationMinutes INT NOT NULL,
	CONSTRAINT pk_service PRIMARY KEY 
		(ServiceId)
);
\end{lstlisting}
\item  Erstellen der Beispieldaten:
\begin{lstlisting}[frame=single, language=SQL, style=SQL, caption={Erstellen der Beispielsdaten}]  
INSERT INTO Genre (GenreId, ParentGenreId, Name)
VALUES (1, NULL, 'Rock'),
  (2, 1, 'Punk Rock'),
  (3, NULL, 'Hip-Hop')
  
INSERT INTO Album
        (AlbumKey, ArtistName, AlbumName, GenreFk)
VALUES 
  (1, 'Red Hot Chilli Peppers', 
    'Californication', 1),
  (2, 'The Ramones', 'Rocket To Russia', 2),
  (3, 'The Beastie Boys', 'Ill Communication', 3)
  
INSERT INTO Track
        (AlbumFk, TrackId, Name, DurationSeconds)
VALUES (3, 1, 'Sure Shot', 200),
  (1, 1, 'Around the World', 239),
  (2, 2, 'Rockaway Beach', 127),
  (1, 4, 'Otherside', 255)
\end{lstlisting}
\item So werden die Tabellen erstellt.
\begin{lstlisting}[frame=single, language=SQL, style=SQL, caption={Erstellen der Tabelle Bestellung und BestellteBier}]  
CREATE TABLE Bestellung (
	BestellungId INT NOT NULL,
	KundenFk INT NOT NULL,
	BarFk INT NOT NULL,
	Zeit DATETIME,
	CONSTRAINT pk_bestellung PRIMARY KEY
		(BesellungId),
	CONSTRAINT fk_kunde FOREIGN KEY(KundenFk)
	REFERENCES Kunde(KundenId),
	CONSTRAIN fk_bar FOREIGN KEY(BarFk)
	REFERENCES Bar(BarId)
);
CREATE TABLE BestellteBier (
	BestellungFk INT NOT NULL,
	SortimentFk INT NOT NULL,
	Anzahl INT NOT NULL,
	CONSTRAINT pk_bestellte_bier PRIMARY KEY
		(BesellungFk, SortimentFk),
	CONSTRAINT fk_bestellung FOREIGN KEY
	  (BestellungFk) REFERENCES 
	  Bestellung(BestellungId),
	CONSTRAINT fk_sortiment FOREIGN KEY
	(SortimentFk)REFERENCES 
	 Sortiment(SortimentId)
);
\end{lstlisting}
\item Die Daten werden wie folgt eingef"ugt:
\begin{lstlisting}[frame=single, language=SQL, style=SQL, caption={Einf"ugen der Bestellung}]  
INSERT INTO Bestellung
  (BestellungId, KundenId, BarId, Zeit)
 VALUES (6, 2, 3, '2015-1-1 02:00')
INSERT INTO BestellteBier
  (BestellungFk, SortimentFk, Anzahl)
VALUES (6, 14, 2)
\end{lstlisting}

\end{enumerate}
\section{Kapitel 3}
\begin{enumerate}
\item Die Abfrage sieht wie folgt aus:
\begin{lstlisting}[frame=single, language=SQL, style=SQL, caption={Abfrage zu Tabelle~\ref{tbl:exbySolution}}]  
SELECT LOWER(Name) AS NameLower,
  DurationMinutes*60*1000 AS DurationMilliseconds
FROM Service
\end{lstlisting}
\item Buchungen sortiert:
\begin{lstlisting}[frame=single, language=SQL, style=SQL, caption={Buchungen sortiert}]  
SELECT *
FROM Booking
ORDER BY StartDate DESC
\end{lstlisting}
\item Alben sortiert:
\begin{lstlisting}[frame=single, language=SQL, style=SQL, caption={Alben sortiert}]  
SELECT *
FROM Album
ORDER BY GenreFk, ArtistName DESC
\end{lstlisting}
\item Termine gefiltert:
\begin{lstlisting}[frame=single, language=SQL, style=SQL, caption={Termine gefiltert}]  
SELECT *
FROM Booking
WHERE BookingStart > '2014-11-24' 
  AND BookingStart < '2014-12-01'
\end{lstlisting}
\item Termine, welche l"anger als 30 Minuten dauern:
\begin{lstlisting}[frame=single, language=SQL, style=SQL, caption={Termine, welche l"anger als 30 Minuten dauern}]  
SELECT *
FROM Booking
WHERE ServiceFk IN (
  SELECT ServiceId
  FROM Service
  WHERE DurationMinutes > 30
)
\end{lstlisting}
\item Erweiterte Tabelle:
\begin{lstlisting}[frame=single, language=SQL, style=SQL, caption={Erweiterte Tabelle}]  
SELECT c.FirstName, c.LastName, s.Name, b.StartDate, 
  DATEADD(minute, s.DurationMinutes, 
    b.StartDate) AS EndDate
FROM Customer c 
  INNER JOIN Booking b 
    ON b.CustomerFk = c.CustomerId
  INNER JOIN Service s 
    ON b.ServiceFk = s.ServiceId
\end{lstlisting}
\item Selbst-JOIN:
\begin{lstlisting}[frame=single, language=SQL, style=SQL, caption={Erweiterte Tabelle}]  
SELECT g1.Name AS SubGenreName, g2.Name AS ParentGenreName
FROM Genre g1
  INNER JOIN Genre g2
    ON g1.ParentGenreFk = g2.GenreId
\end{lstlisting}
\item Die vier Tabellen sehen wie folgt aus:
\begin{table}[H]
         \begin{tabular}{|ll|}
    \hline
    SubGenreName & GenreName     \\ \hline
    Punk Rock & Rock\\    
     \hline
    \end{tabular}
    \caption {INNER JOIN}
        \begin{tabular}{|ll|}
    \hline
    SubGenreName & GenreName     \\ \hline
    Rock & NULL\\
    Punk Rock & Rock\\
    Hip-Hop & NULL\\    
     \hline
    \end{tabular}
    \caption {LEFT JOIN}
          \begin{tabular}{|ll|}
    \hline
    SubGenreName & GenreName     \\ \hline
    Punk Rock & Rock\\
    NULL & Punk Rock\\
    NULL & Hip-Hop\\    
     \hline
    \end{tabular}
    \caption {RIGHT JOIN}
          \begin{tabular}{|ll|}
    \hline
    SubGenreName & GenreName     \\ \hline
    Rock & NULL\\
    Hip-Hop & NULL\\  
    Punk Rock & Rock\\
    NULL & Punk Rock\\
    NULL & Hip-Hop\\
     \hline
    \end{tabular}
    \caption {FULL OUTER JOIN}    
\end{table}
\item Aggregierte Albuml"ange:
\begin{lstlisting}[frame=single, language=SQL, style=SQL, caption={Albuml"ange}]  
SELECT AlbumFk, 
  SUM(DurationMinutes) AS AlbumLength
FROM Tracks
GROUP BY AlbumFk
\end{lstlisting}
\item Mit Albumnamen:
\begin{lstlisting}[frame=single, language=SQL, style=SQL, caption={Albuml"ange mit Albumnamen}]  
SELECT a.ArtistName, a.AlbumName, 
  albumLengths.AlbumLength
FROM Album a INNER JOIN
  ( SELECT AlbumFk, 
      SUM(DurationMinutes) AS AlbumLength
    FROM Tracks
    GROUP BY AlbumFk) albumLengths
  ON a.ArtistId = albumLengths.ArtistFk
\end{lstlisting}
\item HAVING-Filter:
\begin{lstlisting}[frame=single, language=SQL, style=SQL, caption={Albuml"ange mit HAVING-Filter}]  
SELECT AlbumFk, 
   SUM(DurationMinutes) AS AlbumLength
FROM Tracks
GROUP BY AlbumFk
HAVING AVG(DurationMinutes) < 180
\end{lstlisting}
\item Zur Repetition: Eine WHERE-Klausel ist ein Filter f"ur die Daten vor der Aggregation. Die HAVING-Klausel
filtert Daten, nachdem sie aggregiert wurden.
\begin{itemize}
\item Die Anzahl Buchungen ist ein aggregierter Wert. Daher steht diese Bedingung in der HAVING-Klausel.
\item Nach Service muss vor der Aggregation, also im WHERE-Teil gefiltert werden.
\item HAVING-Klausel
\item WHERE-Klausel. Zuerst suchen wir die ersten f"unf Tracks des Albums heraus. Der Durchschnitt wird
anschliessend berechnet.
\end{itemize}
\item Bonustrack-Abfrage:
\begin{lstlisting}[frame=single, language=SQL, style=SQL, caption={Bonustrack-Abfrage}]  
SELECT AlbumFk
FROM Tracks
GROUP BY AlbumFk
HAVING COUNT(*) < MAX(TrackId)
\end{lstlisting}
\item Tabelle Bar:
\begin{lstlisting}[frame=single, language=SQL, style=SQL, caption={Tabelle Bar}]  
SELECT *
FROM Bar
\end{lstlisting}

\item Tabelle Bier:
\begin{lstlisting}[frame=single, language=SQL, style=SQL, caption={Bier}]  
SELECT 
  LOWER(Name) AS NameLower,
  Deziliter * 10 AS Milliliter,
  Einkaufpreis / 1.2 as EinkaufpreisEuro
FROM Bier
\end{lstlisting}

\item Flaschengr"ossen:
\begin{lstlisting}[frame=single, language=SQL, style=SQL, caption={Flaschengr"ossen}]  
SELECT DISTINCT Deziliter
FROM Bier
\end{lstlisting}
\item Sortiment nach Bier:
\begin{lstlisting}[frame=single, language=SQL, style=SQL, caption={Sortiment nach Bier}]  
SELECT *
FROM Sortiment
ORDER BY BierFk
\end{lstlisting}
\item Bier mit Umlaut:
\begin{lstlisting}[frame=single, language=SQL, style=SQL, caption={Bier mit Umlaut}, escapechar=\$]  
SELECT *
FROM Bier
WHERE Name LIKE '%$"a$%'
  OR Name LIKE '%$"o$%'
  OR Name LIKE '%$"u$%'
\end{lstlisting}
\item Biere der Brauerei 1:
\begin{lstlisting}[frame=single, language=SQL, style=SQL, caption={Biere der Brauerei 1}, escapechar=\$]  
SELECT *
FROM Bier
WHERE BrauereiFk = 1
\end{lstlisting}
\item G"unstige Biere:
\begin{lstlisting}[frame=single, language=SQL, style=SQL, caption={G"unstige Biere}, escapechar=\$]  
SELECT *
FROM Bier
WHERE Einkaufspreis / Deziliter < 0.40
\end{lstlisting}
\item Kunde mit Lieblingsbier:
\begin{lstlisting}[frame=single, language=SQL, style=SQL, caption={Kunde mit Lieblingsbier}, escapechar=\$]  
SELECT k.Name, b.Name
FROM Kunde k LEFT JOIN Bier b 
  ON k.LieblingsbierFk = b.BierId
\end{lstlisting}
\item Kunde mit Lieblingsbier sowie unfavorisierte Biere:
\begin{lstlisting}[frame=single, language=SQL, style=SQL, caption={Kunde mit Lieblingsbier sowie unfavorisierte Biere}, escapechar=\$]  
SELECT k.Name, b.Name
FROM Kunde k FULL OUTER JOIN Bier b 
  ON k.LieblingsbierFk = b.BierId
\end{lstlisting}
\item Bestellte Bier in lesbarer Form:
\begin{lstlisting}[frame=single, language=SQL, style=SQL, caption={Bestellte Bier in lesbarer Form}, escapechar=\$]  
select k.Name, bi.Name, b.Zeit
FROM BestellteBier bb
  INNER JOIN Bestellung b 
    ON bb.BestellungFk = b.BestellungId
  INNER JOIN Kunde k
    ON  b.KundenFk = k.KundenId
  INNER JOIN Sortiment s
    ON bb.SortimentFk = s.SortimentId
  INNER JOIN Bier bi
    ON s.BierFk = bi.BierId
\end{lstlisting}
\item Diese Abfrage ist etwas komplizierter. Hier Joinen wir mit einer Unterabfrage. Die Unterabfrage gruppiert die Bestellungen nach Gesamtpreis. 
Dazu m"ussen zwei Spalten aus zwei verschiedenen Tabellen zusammen verrechnet und dann aggregiert werden. Dies ist ohne weitere
Unterabfrage m"oglich.

Die �ussere Abfrage sammelt weitere Informationen zur Bestellung.

Die Innere 
\begin{lstlisting}[frame=single, language=SQL, style=SQL, caption={Bestellungen}, escapechar=\$]  
SELECT b.KundenFk, b.BarFk, bbGrouped.TotalPrice
FROM Bestellung b INNER JOIN
  (SELECT bb.BestellungFk, 
    SUM(bb.Anzahl * s.Preis) AS TotalPrice
  FROM BestellteBier bb
    INNER JOIN Sortiment s
      ON s.SortimentId = bb.SortimentFk
  GROUP BY bb.BestellungFK
  ) bbGrouped
  ON bbGrouped.BestellungFk = b.BestellungId
\end{lstlisting}
\item Wir entdecken, das wir die letzte Aufgabe gr"ostenteils wiederverwenden k"onnen. Die Abfrage wird nun noch etwas komplizierter:
\begin{lstlisting}[frame=single, language=SQL, style=SQL, caption={Umsatz pro Bar}, escapechar=\$]  
SELECT b.BarFk, 
  SUM(bbGrouped.TotalPrice) AS Neujahrsumsatz
FROM Bestellung b INNER JOIN
  (SELECT bb.BestellungFk, 
    SUM(bb.Anzahl * s.Preis) AS TotalPrice
  FROM BestellteBier bb
    INNER JOIN Sortiment s
      ON s.SortimentId = bb.SortimentFk
  GROUP BY bb.BestellungFK
  ) bbGrouped
  ON bbGrouped.BestellungFk = b.BestellungId
WHERE b.Zeit > '2014-31-12 18:00' 
  AND b.Zeit < '2015-01-01 06:00'
GROUP BY b.BarFK
\end{lstlisting}

\item Und wieder k"onnen wir die vorhergehende Abfrage wiederverwenden:
\begin{lstlisting}[frame=single, language=SQL, style=SQL, caption={Trinkfreudige Kunden}, escapechar=\$]  
SELECT k.Name
FROM Kunde k
INNER JOIN
( SELECT b.KundenFk, 
    SUM(bbGrouped.TotalPrice) AS Neujahrsausgaben
  FROM Bestellung b INNER JOIN
    (SELECT bb.BestellungFk, 
      SUM(bb.Anzahl * s.Preis) AS TotalPrice
    FROM BestellteBier bb
      INNER JOIN Sortiment s
        ON s.SortimentId = bb.SortimentFk
    GROUP BY bb.BestellungFk
    ) bbGrouped
  ON bbGrouped.BestellungFk = b.BestellungId
  WHERE b.Zeit > '2014-31-12 18:00' 
    AND b.Zeit < '2015-01-01 06:00'
  GROUP BY b.KundenFk) ka
ON ka.KundenFk = k.KundenId
WHERE ka.Neujahrsausgaben > 20
\end{lstlisting}
\item Dies ist wieder ein einfacher JOIN mit einem GROUP BY
\begin{lstlisting}[frame=single, language=SQL, style=SQL, caption={G"unstigster Preis}, escapechar=\$]
SELECT b.Name, billigBier.BilligPreis
FROM Bier b
INNER JOIN
  (SELECT BierFk, MIN(Preis) AS BilligPreis
  FROM Sortiment
  WHERE Deleted = 0
  GROUP BY BierFk) billigBier
  ON billigBier.BierFk = b.BierId
\end{lstlisting}
\end{enumerate}

\section{Kapitel 4}
\begin{enumerate}
\item L"oschen:
\begin{lstlisting}[frame=single, language=SQL, style=SQL, caption={L"oschen von Eintr"agen}, escapechar=\$]
DELETE 
FROM Tracks
WHERE Name LIKE '% '
\end{lstlisting}

\item L"oschen von 'Various Artists:
\begin{lstlisting}[frame=single, language=SQL, style=SQL, caption={L"oschen von Eintr"agen}, escapechar=\$]
DELETE t
FROM Tracks t INNER JOIN
  Album a ON a.AlbumKey = t.AlbumFk
WHERE a.ArtistName = 'Various Artists'
\end{lstlisting}
\item FullName:
\begin{lstlisting}[frame=single, language=SQL, style=SQL, caption={Einf"again eines Kommentars}, escapechar=\$]
UPDATE Customer c
SET c.FullName = CONCAT(FirstName, ' ', LastName)
\end{lstlisting}

\item In MySql:
\begin{lstlisting}[frame=single, language=SQL, style=SQL, caption={Einf"again eines Kommentars}, escapechar=\$]
UPDATE Booking b
  INNER JOIN Customer c
    ON c.CustomerKey = b.CustomerFk
  INNER JOIN Service s
   ON s.ServiceKey = b.ServiceFk
SET b.Comment = CONCAT(c.FirstName, ' ',
    c.LastName, ' ', s.Name)
\end{lstlisting}

\item Enddatum in MySQL:
\begin{lstlisting}[frame=single, language=SQL, style=SQL, caption={Einf"ugen des Enddatums}, escapechar=\$]
UPDATE Booking b
  INNER JOIN Service s
   ON s.ServiceKey = b.ServiceFk
SET b.startDate = 
  DATEADD(minute, s.DurationMinutes, b.StartDate)
\end{lstlisting}

\item L"oschen:
\begin{lstlisting}[frame=single, language=SQL, style=SQL, caption={L"oschen}, escapechar=\$]
DELETE FROM BestellteBier
WHERE Anzahl > 1
\end{lstlisting}

\item L"oschen von zusammenh"angenden Eintr"agen:
\begin{lstlisting}[frame=single, language=SQL, style=SQL, caption={L"oschen}, escapechar=\$]
DELETE b, bb FROM BestellteBier bb
  INNER JOIN Bestellung b
    ON b.BestellungId = bb.BestellungFk
WHERE b.BestellungId = 4
\end{lstlisting}
\item Um die Referenzielle Integrit"at zu bewahren, darf ein Bier nur dann gel"oscht werden, wenn es nicht referenziert wird.
In diesem konkreten Fall gibt es zwei Tabellen, welche das Bier referenzieren: Kunden und Sortiment. Dies heisst, dass
das Bier nicht das Lieblingsbier eines Kunden sein darf, und auch in keinem Sortiment vorkommen darf. Da Sortimentseintr"age
in dieser Datenbank nicht gel"oscht werden, heisst das, dass das Bier nie in einem Sortiment vorhanden war.
 
 \item Update der Bestellung:
 \begin{lstlisting}[frame=single, language=SQL, style=SQL, caption={Update der Bestellung}, escapechar=\$]
UPDATE BestellteBier
SET Anzahl = 2
WHERE BestellungId = 5
\end{lstlisting}

 \item R"uckwirkendes Update der Preise:
 \begin{lstlisting}[frame=single, language=SQL, style=SQL, caption={R"uckwirkendes update der Preise}, escapechar=\$]
UPDATE Sortiment
SET Preis = Preis + 0.5
WHERE BarFk = 2
\end{lstlisting}

 \item Update der Preise:
 \begin{lstlisting}[frame=single, language=SQL, style=SQL, caption={R"uckwirkendes update der Preise}, escapechar=\$]
UPDATE Sortiment
SET Deleted = 1
WHERE BarFk = 2

INSERT INTO Sortiment 
  (BarFk, BierFk, Preis, Deleted)
SELECT(BarFk, BierFk, Preis + 0.50, 0)
FROM Sortiment
WHERE BarFk = 2
\end{lstlisting}


\end{enumerate}
\end{document}
